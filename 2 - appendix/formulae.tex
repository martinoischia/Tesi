\chapter{Formulae}
\section{Woodbury decomposition}
\label{sec:wood}
The following matrix identity holds.
\begin{proposition}[Woodbury matrix identity]
	Let $M$ be a square $m\times m$ matrix which can be written as the sum $E+UCV$, with E being $m \times m$, $U$ being $m\times n$, $C$ being a square $n\times n$ matrix, and $V$ $n\times m$. Then
	\begin{equation}
		\label{wood}
		M^{-1}=
		\left(E + UCV \right)^{-1} = E^{-1} - E^{-1}U\left(C^{-1} + V E^{-1}U\right)^{-1}V E^{-1}
	\end{equation}
\end{proposition}

In particular, in the case where inverting matrix $E$ can be considered cheap
or useful, on the righ-hand side solving the system involves solving an
$n\times n$ system (the one carachterized by matrix $C^{-1} + V E^{-1}U$),
whilst on the left the dimensions are $m\times m$.\\ This equation is exploited
for faster system solving in the \verb|fdaPDE| library. As an example, consider
the space-only problem, described \eg in \cite{sangalli1}: the presence of
covariates leads to a linear system involving the following matrix $M$:
\begin{equation*}
	M =
	\begin{bmatrix}
		\Psi^TQ\Psi  & -\lambda R_1^T \\
		-\lambda R_1 & -\lambda R_0
	\end{bmatrix}
\end{equation*} Remembering that the projection matrix Q is defined as $I-H$, $M$
can be split into the following two components, one independent from $\lambda$:
\begin{equation*}
	M =
	\begin{bmatrix}
		\Psi^T\Psi   & -\lambda R_1^T \\
		-\lambda R_1 & -\lambda R_0
	\end{bmatrix}
	+
	\begin{bmatrix}
		-\Psi^TH\Psi & 0 \\
		0            & 0
	\end{bmatrix}
\end{equation*} By defining $E$ the left matrix of the two above (which is also the
matrix corresponding to the problem without covariates), and remembering that
$H = W\left(W^TW\right)^{-1}W^T$, Woodbury decomposition can be exploited
defining the following matrices $U, C, V$:
\begin{equation*}
	\begin{bmatrix}
		-\Psi^TW\left(W^TW\right)^{-1}W^T\Psi & 0 \\
		0                                     & 0
	\end{bmatrix}
	= UCV
\end{equation*}
\begin{equation*}
	U =
	\begin{bmatrix}
		\Psi^TW \\
		0
	\end{bmatrix}
\end{equation*}
\begin{equation*}
	C = -
	\begin{bmatrix}
		\left(W^TW\right)^{-1}
	\end{bmatrix}
\end{equation*}
\begin{equation*}
	V = U^T =
	\begin{bmatrix}
		W^T\Psi & 0
	\end{bmatrix}
\end{equation*} where the $0s$ indicate matrices of zeros of suitable dimensions.\\
Since matrices $U, C, V$ do not depend on $\lambda$, computing the solution of
the system for different values of $\lambda$ involves only the factorization of
the matrix $E$ (and the cheaper $C^{-1} + V E^{-1}U$).