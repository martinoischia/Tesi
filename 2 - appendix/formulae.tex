% !TEX root = ../thesis.tex
\chapter{Formulae}
\section{Woodbury decomposition}
\label{sec:wood}
The following matrix identity holds.
\begin{proposition}[Woodbury matrix identity]
	Let $M$ be a square $m\times m$ matrix which can be written as the sum $E+UCV$, with E being $m \times m$, $U$ being $m\times n$, $C$ being a square $n\times n$ matrix, and $V$ $n\times m$. Then
	\begin{equation}
		\label{wood}
		M^{-1}=
		\left(E + UCV \right)^{-1} = E^{-1} - E^{-1}U\left(C^{-1} + V E^{-1}U\right)^{-1}V E^{-1}.
	\end{equation}
\end{proposition}

Woodbury decomposition is useful in the case where computing a factorization of
matrix $E$ can be considered cheap or advantageous, \ie solving a system
involving this matrix is necessary in many places. In fact, looking at the
right hand side of previous equation, solving a system involving matrix $M$ can
be done by solving:
\begin{itemize}
	\item[--]several $m \times m$ systems involving matrix $E$;
	\item[--]an $n \times n$ system, where $n$ is supposed to be smaller than $m$, characterized by matrix
	$ \left( C^{-1} + V E^{-1}U \right)$, as well as several $n \times n$ systems to compute the
	inverse of $C$, but we will see it is not necessary in our case.
\end{itemize}
This equation is exploited for faster system solving in the
\verb|fdaPDE| library.\newline As an example, consider the space-only problem,
described \eg in \cite{sangalli1}: the presence of covariates leads to a linear
system involving the following matrix $M$:
\begin{equation*}
	M =
	\begin{bmatrix}
		\Psi^TQ\Psi  & -\lambda R_1^T \\
		-\lambda R_1 & -\lambda R_0
	\end{bmatrix}
	.
\end{equation*} Remembering that the projection matrix Q is defined as $I-H$, $M$
can be split into the following two components, one independent from $\lambda$:
\begin{equation*}
	M =
	\begin{bmatrix}
		\Psi^T\Psi   & -\lambda R_1^T \\
		-\lambda R_1 & -\lambda R_0
	\end{bmatrix}
	+
	\begin{bmatrix}
		-\Psi^TH\Psi & 0 \\
		0            & 0
	\end{bmatrix}
	.
\end{equation*} By defining $E$ the left matrix of the two above (which is also the
matrix corresponding to the problem without covariates), and remembering that
$H = W\left(W^TW\right)^{-1}W^T$, Woodbury decomposition can be exploited, by
defining the matrices $U, C, V$ in the following way:
\begin{equation*}
	UCV=
	\begin{bmatrix}
		-\Psi^TW\left(W^TW\right)^{-1}W^T\Psi & 0 \\
		0                                     & 0
	\end{bmatrix}
	,
\end{equation*}
\begin{equation*}
	U =
	\begin{bmatrix}
		\Psi^TW \\
		0
	\end{bmatrix}
	,
\end{equation*}
\begin{equation*}
	C = -
	\begin{bmatrix}
		\left(W^TW\right)^{-1}
	\end{bmatrix}
	,
\end{equation*}
\begin{equation*}
	V = U^T =
	\begin{bmatrix}
		W^T\Psi & 0
	\end{bmatrix}
	,
\end{equation*} where the $0s$ indicate matrices of zeros of suitable dimensions.

In this case solving a system involving matrix $C$ is simply a multiplication
by $W^T W$.

\section{Handling missing values}
\label{secNA}
Consider, as in appendix \ref{sec:wood}, the basic space-only problem,
characterized by equations
\begin{equation}
	\begin{bmatrix}
		\Psi^TQ\Psi  & -\lambda R_1^T \\
		-\lambda R_1 & -\lambda R_0
	\end{bmatrix}
	\begin{bmatrix}
		\bm{f} \\
		\bm{g}
	\end{bmatrix}
	=
	\begin{bmatrix}{\Psi}^T Q\bm{z} \\
		0
	\end{bmatrix}
	.
\end{equation}
We could use the notation $\Psi_n$, $Q_n$, $z_n$ since these are the
quantities that depend on the $n$ observed values.

Now suppose that the $m$-th observation $z_m$ is not available. If we were
unaware of this information, we would build the system with quantities of index
$n-1$:

\begin{equation}
	\label{n-1}
	\begin{bmatrix}
		\Psi_{n-1}^TQ_{n-1}\Psi_{n-1} & -\lambda R_1^T \\
		-\lambda R_1                  & -\lambda R_0
	\end{bmatrix}
	\begin{bmatrix}
		\bm{f} \\
		\bm{g}
	\end{bmatrix}
	=
	\begin{bmatrix}{\Psi}_{n-1}^T Q_{n-1}\bm{z}_{n-1} \\
		0
	\end{bmatrix}
\end{equation}

From a computational point of view, when implementing the mixed-effects model
with multiple systems to solve analogous to the previous one, or simililarly in
spatio-temporal regression, it is certainly preferable to keep the matrices
with the original size rather than building matrices of different sizes. This
fact justifies the need to write a system equivalent to \ref{n-1} as a system
corresponding to an $n$ observations case.\\ We impose equality of
correspondant terms.\\ By definition of $Q=I-W \left( W^T W \right) W^T$, it is
logical to set to $0$ the covariates corresponding to the $m$-th unused
observation: in such a way $W^T_nW_n=W^T_{n-1}W_{n-1}$. This operation leads to
$Q_n$ being the matrix obtained from $Q_{n-1}$ by adding the $m$-th row and the
$m$-th column of zeros except for the unitary diagonal element.\\ For the
left-hand side,
\begin{equation}
	\left(\Psi^T_nQ_n\Psi_n\right)_{ij}=\left(\Psi_{n-1}^TQ_{n-1}\Psi_{n-1}\right)_{ij}
\end{equation}
leads to
\begin{equation}
	\sum_{k,l}\left(\Psi_n\right)_{ki}\left( Q_n \right)_{kl} \left( \Psi_n\right)_{lj}=
	\sum_{k,l}\left(\Psi_{n-1}\right)_{ki}\left( Q_{n-1} \right)_{kl} \left( \Psi_{n-1}\right)_{lj}.
\end{equation}
Removing the identical terms, we are left with:
\begin{equation}
	\sum_{k=m,l}\left(\Psi_n\right)_{ki}\left( Q_n \right)_{kl} \left( \Psi_n\right)_{lj}
	+
	\sum_{k \neq m,l=m}\left(\Psi_n\right)_{ki}\left( Q_n \right)_{kl} \left( \Psi_n\right)_{lj}
	=0
\end{equation}
Removing the zero terms, we obtain
\begin{equation}
	\left(\Psi_n\right)_{mi}\left( \Psi_n\right)_{mj}=0
\end{equation}
We obtain then that $\Psi_n$ is the same as $\Psi_{n-1}$ except for
the $m$-th row, where its original value is replaced by zeros. Recalling the
meaning of $\Psi$ matrix, this means that any function evaluates to $0$ in the
position corresponding to the missing value.

Comparing the right-hand side,
\begin{equation}
	\left( \Psi_{n-1}^T Q_{n-1}\bm{z}_{n-1}\right)_i =
	\left( \Psi_{n}^T Q_{n}\bm{z}_{n}\right)_i
\end{equation}
we get
\begin{equation}
	\sum_{k,l} \left( \Psi_{n}\right)_{ki} \left(  Q_{n} \right)_{kl} \left( \bm{z}_{n} \right)_l
	=
	\sum_{k,l} \left( \Psi_{n-1}\right)_{ki} \left(  Q_{n-1} \right)_{kl} \left( \bm{z}_{n-1} \right)_l .
\end{equation}
Again, assuming we set $z_m=0$, the differences come on lhs when
$k=m$ or $l=m$. But the case $l=m$ is 0 because $z_m$ is $0$. The case $k=m$ is
0 because $\left( Q_n\right)_{ml}$ is 0.

In conclusion, for writing an equivalent system it is sufficient, like one
could naturally guess, to:
\begin{itemize}
	\item[--] replace the missing value with $0$;
	\item[--] replace the covariates corresponding to the missing value with $0$s;
	\item[--] replace the row of $\Psi$ matrix corresponding to the missing value with $0$s.
\end{itemize}
Notice that it is straightforward to generalize this result in the
case where the missing values are more than one.
