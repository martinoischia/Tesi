% !TEX root = ../thesis.tex
\chapter{Introduction}
\label{ch:intro}

\section{Spatial regression with partial differential equation regularization}
This work naturally stems from the studies in the research field called
\textit{spatial regression with partial differential equation regularization},
abbreviated in the following as SR-PDE. SR-PDE constitutes a family of models
which has been, and still is, under study beginning from \citeyear{sangalli0}
in \citeauthor{sangalli0} \cite{sangalli0}.

SR-PDE effectively combines several branches of mathematics and statistics
modeling. For a thorough analysis of its derivation and the enhancements it
brings to the existing literature see \cite{sangalli1}, I will limit myself to
a brief introduction.\\ SR-PDE models feature the minimization of a functional
composed of two terms: a classical least-square term, with the goal of
estimating a vector of unknown regression coefficients (in case of presence of
covariates), and a regularization term, estimating an unknown deterministic
spatial field.\\ The spatial field contributes to the functional by integration
over the spatial domain of the square of a differential operator, most commonly
the Laplacian operator. By properties of the Laplace operator, this choice
induces an isotropic and stationary smoothing effect on the unknown spatial
field, meaning equal in every direction and independent on the position. Other
differential operators considered in this work are generic second order
differential operators, that come from the physics of the problem or from some
preexisting knowledge about the phenomenon under study.

The inclusion of partial differential equations in the statistical model brings
many advantages, like the aforementioned ability to include problem-specific
knowledge, dealing with boundary conditions, etc., but makes computations more
cumbersome.

\section{Mixed-effects models}
In a situation where observed data possess a natural grouping structure,
mixed-effects models are often utilized. They combine fixed effects, meaning
that a set of covariates have regression coefficients shared among all groups,
and random effects, meaning the remaining set of covariates have regression
coefficients varying along each group.

Given our interest in spatial data analysis, a typical situation in which
mixed-models are used is in clinical data, where data of several patients are
measured in the same area of the body. For example in chapter
\ref{ch:Applications} \nameref{ch:Applications}, we will see a mixed-effects
model applied to human brain data, collected on a set of patients who undertook
functional Magnetic Resonance Imaging (fMRI).\\ For this reason in the
following we will use the word \textit{patients} to indicate the different
groups of our generic mixed-effects model.

We are therefore ready to present, similarly as in \cite{kim}, the SR-PDE
mixed-effects model.

\section{Generic SR-PDE mixed-effects model}

Consider $m$ patients. To the $i$-th patient, with $i$ varying from 1 to m,
corresponds a unique spatial domain $\Omega_i$, on which we observe $n_i$ data
at different positions $\bm{p}_{ij}$. Index $j$ is therefore varying from $1$
to $n_i$, for given patient $i$.\\The variable of interest z, observed at
$\bm{p}_{ij}$, is modeled as
\begin{equation}
	\label{model}
	z_{ij} = \bm{w}^T_{ij} \bm{\beta} + \bm{v}^T_{ij} \bm{b}_i + f_i(\bm{p_}{ij}) + \epsilon_{ij}.
\end{equation}
The notation used represents the following quantities:
\begin{itemize}
	\item[--] $\bm{w}_{ij}$ is the vector of fixed effects covariates for the
		observation $z_{ij}$; \item[--] $\bm{\beta}$ is the vector of regression
		coefficients for fixed effects; \item[--] $\bm{v}_{ij}$ is the vector of random
		effects covariates for the observation $z_{ij}$; \item[--] $\bm{b}_i$ is the
		vector of regression coefficients of the random effects for patient $i$;
	\item[--] $f_i$ is the unknown deterministic field defined on domain
		$\Omega_i$; \item[--] $\epsilon_{ij}$, for every $i$ and $j$, are the random
		noise or errors, considered as the realizations of independent identically
		distributed (i.i.d.) random variables, with mean $0$ and variance $\sigma^2$.
\end{itemize}
We will assume that the random effect $\bm{b}_i$ has average $0$
\begin{equation}
	\label{constraint}
	\sum_{i=1}^{m}{\bm{b}_i}=0.
\end{equation}
In fact, if the average was different from 0, we would simply split
the contribution of those covariates into two, a fixed effect one and a random
effect covariate with 0 average across groups (more on this in section
\ref{repar}, \nameref{repar}). \\ Let also $N$ be equal to $\sum_{i=1}^{m}n_i$.
For every $i$, $j$ equation \ref{model} constitutes a system of $N$ equations,
that is expressed concisely in algebraic form in the following two ways:
\begin{equation}
	\begin{cases}
		\bm{z}_i = W_i \bm{\beta} + \bm{V}_i \bm{b}_i + \bm{f_i} + \bm{\epsilon_i} \qquad
		i=1\dots m \\
		\sum_{i=1}^{m}{\bm{b}_i}=0
	\end{cases}
\end{equation}
and
\begin{equation}
	\label{unconstrained}
	\bm{z} = W \bm{\beta} + \bm{V} \bm{b} + \bm{f}_N + \bm\epsilon.
\end{equation}
Describing each corresponding term in the two equations:
\begin{itemize}
	\item[--] $\bm{z}_i \in \R^{n_i}$ is the vector of observed data for patient
		$i$ and $\bm{z}$ is the vector belonging to $\R^N$ obtained by concatenating
		the $m$ vectors $\bm{z}_i$; \item[--] $W_i$ is the matrix whose element
		$w_{j,k}$ is the $k$-th element of previously defined vector $\bm{w}_{ij}$,
		$\left(\bm{w}_{ij}\right)_k$, whereas $W$ is a concatenation as well
		\begin{equation}
			W=
			\begin{bmatrix*}
				W_1\\
				\vdots\\
				W_m
			\end{bmatrix*}
			;
		\end{equation}
	\item[--] $V_i$ is the matrix whose element $v_{j,k}$ is the $k$-th
		element of previously defined vector $\bm{v}_{ij}$,
		$\left(\bm{v}_{ij}\right)_k$; \item[--] when moving to the $N$ system of
		equations, we include constraint \ref{constraint} by defining $\bm{b}$ as the
		concatenation of $m-1$ vectors $\bm{b}_i$, omitting the last one
		\begin{equation}
			\bm{b}=
			\begin{bmatrix*}
				\bm{b}_1\\
				\vdots\\
				\bm{b}_{m-1}
			\end{bmatrix*}
		\end{equation}
		and defining $V$ in the following manner:
		\begin{equation}
			V=
			\begin{bmatrix*}
				V_1 & 0 & 0 & 0\\
				0 & V_2 & 0 & 0\\
				\vdots & 0 & \ddots & \vdots\\
				0 & 0 & 0 & V_{m-1}\\
				-V_m & -V_m & -V_m & -V_m
			\end{bmatrix*}
			;
		\end{equation}
	\item[--] $\bm{f}_i \in \R^{n_i}$ is the vector of observed data for
		patient $i$ and $\bm{f}_N$ is the vector belonging to $\R^N$ obtained by
		concatenating the $m$ vectors $\bm{f}_i$; \item[--] $\bm{\epsilon}_i \in
			\R^{n_i}$ is the vector of errors for patient $i$ and $\bm{\epsilon}_N$ is the
		vector belonging to $\R^N$ obtained by concatenating the $m$ vectors
		$\bm{\epsilon}_i$.
\end{itemize}
\section{Reparametrizing the model}\label{repar}
The $N$ system of equations \ref{unconstrained} is the model written in an
unconstrained form, where the constraint \ref{constraint} has been “injected”
into the design matrix. We also call it \textit{official parametrization} of
the mixed-effects model.

As described in \cite{kim} we prefer to adopt a slightly different approach,
mainly for computational efficiency reasons. In the following we will assume
that all covariates contribute to the fixed effect part of the model, whereas a
subset of them will contribute to the random effect part. We will refer to this
version of the model as the \textit{implementation} one.

\textit{Implementation} version is expressed by the following system:
\begin{equation}
	\bm{z}=W' \bm{\beta}' +V'\bm{b}' +\bm{f}_N + \bm{\epsilon},
\end{equation}
where we have used the following new quantities:
\begin{itemize}
	\item[--] $W'$, similarly as before, is concatenation of $W_i'$ matrices, where
		$W_i'$ is composed of the covariates related only to fixed effects, meaning
		with no patient-specific contribution.
		\begin{equation}
			W'=
			\begin{bmatrix*}
				W_1'\\
				W_2'\\
				\vdots\\
				W'_{m}
			\end{bmatrix*}
			;
		\end{equation}
	\item[--] $\bm{\beta}'$ is the coefficient relative to covariates for
		fixed effects, as just described; \item[--] $V'$ is composed of $V_i'$
		matrices, which are the matrices of covariates having also a random effect.
		Unlike previous $V$, $V'$ does not include the part deriving from constraint
		\ref{constraint}:
		\begin{equation}
			V'=
			\begin{bmatrix*}
				V_1' & 0 & 0 & 0\\
				0 & V_2' & 0 & 0\\
				\vdots & 0 & \ddots & \vdots\\
				0 & 0 & 0 & V_{m}'
			\end{bmatrix*}
			;
		\end{equation}
	\item[--] $\bm{b}'$ is the regression coefficient relative to the
		covariates with patient-specific effect,
		\begin{equation}
			\bm{b}'=
			\begin{bmatrix*}
				\bm{b}'_1\\
				\vdots\\
				\bm{b}'_{m}
			\end{bmatrix*}
			;
		\end{equation}
\end{itemize}
Let us also use the following notation, which allows us to properly
refer to sizes of vectors and matrices defined before.\\ We indicate with $q$
the number of covariates, and with $p$ the number of covariates with
patient-specific interest. Therefore $q-p$ is the number of covariates
considered for their fixed effect contribution only.

Vector $\bm{b}'$ belongs to $\R^{mp}$, $\bm{\beta}'$ to $\R^{q-p}$ and we can
relate it to previously defined vectors $\bm{b}$, $\bm{\beta}$ ($\in
	\R^{m(p-1)},\R^q$ respectively), by defining $\bm{\beta}^* \in \R^p$, average
of random effects coefficients
\begin{equation}
	\bm{\beta}^*=\frac{\sum_{i=1}^{m}\bm{b}'_i}{m}.
\end{equation}
In this way $\bm{\beta}$ is just the concatenation of $\bm{\beta}'$
and $\bm{\beta}^*$ while for $\bm{b}$ holds the following:
\begin{equation}
	\bm{b}=
	\begin{bmatrix}
		\bm{b}_1 \\
		\bm{b}_2 \\
		\vdots   \\
		\bm{b}_{m-1}
	\end{bmatrix}
	=
	\begin{bmatrix}
		\bm{b}_1' -\bm{\beta}^* \\
		\bm{b}_2' -\bm{\beta}^* \\
		\vdots                  \\
		\bm{b}_{m-1}' -\bm{\beta}^*
	\end{bmatrix}
	.
\end{equation}
\section{Estimation problem}
Similarly to what happens with every SR-PDE model, we estimate the unknown
quantities of the model by solving, with the necessary approximations, the
following minimization problem:
\begin{equation}
	\label{problem}
	\textrm{Find} \argmin_{\bm{\beta}', \bm{b}', f_1, \dots, f_m}{J_{\Omega_i, \lambda} \left(\bm{\beta}', \bm{b}', f_1, \dots, f_m \right)}
\end{equation}
where the functional $J_{\Omega_i, \lambda}$ is defined as
\begin{equation}
	\label{functional}
	J_{\Omega_i, \lambda} \left(\bm{\beta}', \bm{b}', f_1, \dots, f_m \right) =
	\norm{\bm{z}-W' \bm{\beta}' -V'\bm{b}' -\bm{f}_N}^2 + \lambda \sum_{i = 1}^m \int_{\Omega_i}\Delta f_i \left(\bm{p}\right)^2 d\Omega_i ,
\end{equation}
or equivalently, in a more lengthy expression:
\begin{multline}
	J_{\Omega_i, \lambda} \left(\bm{\beta}', \bm{b}'_1, \dots, \bm{b}'_m, f_1, \dots, f_m \right) = \\ \sum_{i = 1}^m \left( \sum_{j=1}^{n_i} \left( z_{ij}-{\bm{w}'_{ij}}^T \bm{\beta}' - {\bm{v}'_{ij}}^T \bm{b}'_i - f_i(\bm{p_{ij}}) \right)^2 + \lambda \int_{\Omega_i} \Delta f_i \left(\bm{p}\right)^2 d\Omega_i\right).
\end{multline}
For simplicity, we have considered the Laplacian as differential
operator, but more general choices are possible.

It shall be noticed that the unknown field must not satisfy the differential
equation but contributes to the functional with the square of its misfit from
the equation itself --- besides its contribution in terms of distance from
observed data.

We now introduce the notation necessary to tackle problem \ref{problem}, and
characterize the solution defined on suitable spaces. Assuming $W'_i$ and
$V'_i$ for $i=1,\dots,m$ full rank, we define the following matrices:
\begin{equation}
	\label{matrix:x}
	X =
	\begin{bmatrix}
		W'_1     & V'_1   & 0      & \ldots & 0        & 0      \\
		W'_2     & 0      & V'_2   & \ldots & 0        & 0      \\
		\vdots   & \vdots & \vdots & \ddots & \vdots   & \vdots \\
		W'_{m-1} & 0      & 0      & \ldots & V'_{m-1} & 0      \\
		W'_m     & 0      & 0      & \ldots & 0        & V'_m
	\end{bmatrix}
\end{equation}
\begin{equation}
	H = X\left(X^TX\right)^{-1}X^T
\end{equation}
\begin{equation}
	Q = I_N - H
\end{equation}
with $X\in \R^{N\times (mp+q-p)}$, H and Q $\in \R^{N\times N}$ are
the matrices that project a vector, respectively, onto the subspace spanned by
the columns of X and onto its orthogonal complement with respect to $\R^N$.
Notice that despite the matrix $X^TX$ exhibits the pattern in
figure~\ref{fig:pattern1},
\begin{figure}[t]
	\begin{subfigure}{0.45\textwidth}
		\includegraphics[width=\textwidth]{images/3pattern.png}
		\caption{}
		\label{fig:pattern1}
	\end{subfigure}
	\hfill
	\begin{subfigure}{0.45\textwidth}
		\includegraphics[width=\textwidth]{images/full_pattern.png}
		\caption{}
		\label{fig:pattern2}
	\end{subfigure}
	\caption{\textit{(\subref{fig:pattern1}) shows the pattern of
			generic symmetric matrix $X^TX$ as defined in equation~\ref{matrix:x}. Size has
			been arbitrarily chosen to be 40. On the right we show the pattern of the
			inverse of such matrix, illustrating that the inverse is dense. Non-zero values
			were sampled from a uniform distribution.}} \label{fig:pattern}
\end{figure}
where only the first row, the first column and the diagonal are
different from 0, the inverse of this type of matrix is in general dense,
\textit{c.f.} figure~\ref{fig:pattern2}.

Define also the vector of coefficients $\bm{\nu} = (\bm{\beta}^\prime,
	\bm{b}_1^\prime, \dots, \bm{b}_m^\prime) \in \R^{mp+q-p}$.\\ Given these
definitions, we can write the mixed-effects model in \textit{implementation}
version with the formula
\begin{equation}
	\label{modelX}
	\bm{z} = X \bm{\nu} + \bm{f}_N + \bm{\epsilon},
\end{equation}
which separates the two components of our model, the parametric one
and the nonparametric. The estimation functional \ref{functional} can then be
expressed as
\begin{equation}
	\label{functional_short}
	J_{\Omega_i, \lambda} \left(\bm{\nu} , f_1, \dots, f_m \right) =
	\norm{\bm{z}-X\bm{\nu}  -\bm{f}_N}^2 + \lambda \sum_{i = 1}^m \int_{\Omega_i}\Delta f_i \left(\bm{p}\right)^2 d\Omega_i ,
\end{equation}
Next section deals with the characterization of a possible minimizer
$\left(\hat{\bm{\nu}} , \hat{f}_1, \dots, \hat{f}_m \right)$ of functional
\ref{functional_short}.
\section{Solving the estimation problem}
Assuming that for each patient $i$, spatial field $f_i$ belongs to
$\mathcal{H}^2(\Omega_i)$, Sobolev space of functions whose first and second
derivative are in $\mathcal{L}^2(\Omega_i)$, the functional
\ref{functional_short} is well-defined.\\ We also make the assumption, not
strictly necessary, of imposing on every field $f_i$ homogeneous boundary
condition of Neumann type, meaning the normal derivative on the boundary is
null almost everywhere. Interesting applications of different type of boundary
conditions in the context of SR-PDE models are treated in \cite{Azzimonti}.

The field $f= \left(f_1 \dots f_m \right) $ is therefore naturally set in the
Hilbert space
\begin{equation}
	\mathcal{V}=\bigoplus_{i=1}^m \mathcal{H}^2_{n0}(\Omega_i)
\end{equation}
where $\mathcal{H}^2_{n0}(\Omega_i)$ is the set $\left\{g \in
	\mathcal{H}^2(\Omega_i) \mid \nabla g \cdot \bm{n} = 0 \text{ a.e. on } \partial \Omega_i\right\}$.

For the estimation problem introduced before, it holds the following:
\begin{theorem}
	With $\bm{\nu} \in \R^{mp+q-p}$ and $f =\left(f_1 \dots f_m \right) \in \mathcal{V}$, for functional $J_{\Omega_i, \lambda} \left(\bm{\nu} , f\right)$  there exists one and only one minimizer $\left(\hat{\bm{\nu}} , \hat{f} \right)$.
\end{theorem}

By denoting also with $\bm{\varphi}_N$ the vector obtained by concatenation of
the evaluations of a generic element $\varphi$ belonging to functional space
$\mathcal{V}$, at locations $n_i$ for every $i$, we have the following
characterization for the solution $\left(\hat{\bm{\nu}} , \hat{f} \right) =
	\left(\hat{\bm{\nu}} , \hat{f}_1, \dots, \hat{f}_m \right)$ of the estimation
problem:
\begin{equation}
	\begin{cases}
		\bm{\varphi}_N^T Q \hat{\bm{f}}_N + \lambda \sum_{i = 1}^m \int_{\Omega_i}\Delta \varphi_i \Delta \hat{f}_i  d\Omega_i = \bm{\varphi}_N^T Q \bm{z} \quad \forall \varphi \in \mathcal{V} \\
		\hat{\bm{\nu}}=\left(X^TX\right)^{-1}X^T(\bm{z}-\hat{\bm{f}}_N)
	\end{cases}
	. \label{sys1}
\end{equation}
First of previous equations can be equivalently rewritten into two
equations. With the introduction of an auxiliary function $\hat{g}$, whose
vector components are $-\Delta \hat{f}_i$, we write system \ref{sys1} as
\begin{equation}
	\label{triple}
	\begin{cases}
		\bm{\varphi}_N^T Q \hat{\bm{f}}_N + \lambda \sum_{i = 1}^m \int_{\Omega_i}\nabla \varphi_i \nabla g_i d\Omega_i = \bm{\varphi}_N^T Q \bm{z} \\
		\sum_{i=1}^{m}\left(\int_{\Omega_i}\eta_i \hat{g}_i d\Omega_i +\int_{\Omega_i}\nabla \eta_i \cdot \nabla \hat{f}_i d\Omega_i\right)=0       \\
		\hat{\bm{\nu}}=\left(X^TX\right)^{-1}X^T(\bm{z}-\hat{\bm{f}}_N)
	\end{cases}
	,
\end{equation}
valid for every couple $(\varphi,\eta)$ belonging to appropriate
functional spaces (essentially, we can work here with Sobolev spaces of degree
one).\\ This form of the problem is the most suitable for the use of the finite
element method to compute the desired estimation.

\section{Numerical solution}
In the library \texttt{fdaPDE} we solve the estimation problem by mean of
finite elements (for detailed description see e.g. \cite{quarteronitosto}). We
generate a triangular mesh, by partitioning domains $\Omega_i$ with a regular
triangulation. Domain $\Omega_i$ is therefore approximated as
$\Omega_i^{\tau_i}$, union of the triangles.

On the approximated domains we define Sobolev spaces,
$\mathcal{H}^1_i\left(\Omega_i^{\tau_i}\right)$, for every $i$, and relative
subspaces, $V^1_{i}\left(\Omega_i^{\tau_i}\right)$, \ie the finite element
space of degree one on triangulation $\tau_i$: in this work we will limit
ourselves to Lagrangian finite elements of degree one with nodes in the
vertices of the triangles.

Having defined this spaces, we move to the discrete counterpart of problem
\ref{triple}. The equations remain the same if not for the domain $\Omega_i$ ,
approximated as $\Omega_i^{\tau_i}$. But as we change functional spaces from
infinite dimension to finite dimension:
\begin{itemize}
	\item[--] we can write functions as linear (finite) combinations of finite
		element basis; we denote as $\bm{\psi}_i$ or $\bm{\psi}^{\left(i\right)}$ if
		necessary, the vector of fields that constitute a basis for
		$V^1_{i}\left(\Omega_i^{\tau_i}\right)$; \item[--] we can compute integrals by
		numerical techniques, see \cite{quarteronitosto} for technical
		details;\item[--] thanks to previous two points, problem \ref{triple} is
		equivalently written in algebraic form as a linear system of equations.
\end{itemize}
The following matrices definitions are used:\\
\begin{equation}
	\Psi_i=
	\begin{bmatrix}
		\psi_1^{(i)}\left(\bm{p}_{i1}\right)   & \ldots & \psi_{N_{\tau_i}}^{(i)}\left(\bm{p}_{i1}\right)   \\
		\vdots                                 & \ddots & \vdots                                            \\
		\psi_1^{(i)}\left(\bm{p}_{in_i}\right) & \ldots & \psi_{N_{\tau_i}}^{(i)}\left(\bm{p}_{in_i}\right)
	\end{bmatrix}
	.
\end{equation}
$\Psi_i$ belongs to $\R^{n_i \times N_{\tau_i}}$, where we define
$N_{\tau_i}$ as the number of basis functions deriving from triangulation
$\tau_i$ relative to $\Omega_i$. Given Neumann boundary conditions, this is
also the number of nodes of the $i$-th mesh.
\begin{equation}
	R_{0i}=\int_{\Omega_{\tau_i}} \bm{\psi}_i \bm{\psi}^T_i d\Omega_{\tau_i}.
\end{equation}
$R_{0i}$ lies in $\R^{N_{\tau_i} \times N_{\tau_i}}$.
\begin{equation}
	R_{1i}=\int_{\Omega_{\tau_i}} \left( \nabla \bm{\psi}_i\right)^T \nabla  \bm{\psi}_i d\Omega_{\tau_i}.
\end{equation}
$R_{1i}$ lies in $\R^{N_{\tau_i} \times N_{\tau_i}}$ as well, with
$\nabla \bm{\psi}_i$ being the matrix whose element in $m$-th row and $n$-th
column is $\pdev{\left(\bm{\psi}_i\right)_n}{x_m}$.

Before we used the operator mapping a generic element $\varphi$, belonging to
functional space $\mathcal{V}$, to $\bm{\varphi}_N$, the vector obtained by
concatenation of the evaluations of $\varphi$, at locations $n_i$ for every
$i$. By defining $\tilde{\Psi}$ as
\begin{equation}
	\tilde{\Psi}=
	\begin{bmatrix}
		\Psi_1 &        & 0      \\
		       & \ddots &        \\
		0      &        & \Psi_m
	\end{bmatrix}
	,
\end{equation}
the following identity shows the role of $\Psi_i$ matrices:
\begin{equation}
	\bm{\varphi}_N=\tilde{\Psi}\bm{\varphi}
\end{equation}
where $\bm{\varphi}$ is the coefficient vector of expansion of
$\varphi$ with respect to finite element basis.

Analogously as $\tilde{\Psi}$, we define the tensorised versions of $R_0$ and
$R_1$ matrices as:
\begin{equation}
	\tilde{R_0}=
	\begin{bmatrix}
		R_{01} &        & 0      \\
		       & \ddots &        \\
		0      &        & R_{0m}
	\end{bmatrix}
	,
\end{equation}
\begin{equation}
	\tilde{R_1}=
	\begin{bmatrix}
		R_{11} &        & 0      \\
		       & \ddots &        \\
		0      &        & R_{1m}
	\end{bmatrix}
	.
\end{equation}

By mean of simple algebraic manipulations, we express equivalently the discrete
counterpart of problem \ref{triple} with the following linear system of
equations:

\begin{equation}
	\label{mono}
	\begin{bmatrix}
		\tilde{\Psi}^TQ\tilde{\Psi} & -\lambda \tilde{R}_1^T \\
		-\lambda \tilde{R}_1        & -\lambda \tilde{R}_0
	\end{bmatrix}
	\begin{bmatrix}
		\hat{\bm{f}} \\
		\hat{\bm{g}}
	\end{bmatrix}
	=
	\begin{bmatrix}
		\tilde{\Psi}^T Q\bm{z} \\
		0
	\end{bmatrix}
\end{equation}
together with the least square equation for parameter $\bm{\nu}$,
which can be rewritten as
\begin{equation}
	\label{nu}
	\hat{\bm{\nu}}=\left(X^TX\right)^{-1}X^T(\bm{z}-\tilde{\Psi}\hat{\bm{f}})
\end{equation}
We call system \ref{mono} monolithic because it might be tough to
solve it by standard numerical techniques. In fact number of patients $m$ might
be large, and so the number of nodes $N_{\tau_i}$ for each patient $i$.

Therefore, aim of this work is to avoid the solution of such high dimension
linear system, in favour of splitting it into many systems ($\simeq m$) of
lower dimension.

In case the dimensions of the monolithic system are treatable, the Woodbury
decomposition formula, described in appendix \ref{sec:wood}, can be used to
speed up the computation of the solution for different values of $\lambda$. The
decomposition for system \ref{mono} is analogous to the one described in the
appendix, with
\begin{equation}
	\label{eq:wootilde}
	E =
	\begin{bmatrix}
		\tilde{\Psi}^T\tilde{\Psi} & -\lambda \tilde{R}_1^T \\
		-\lambda \tilde{R}_1       & -\lambda \tilde{R}_0
	\end{bmatrix}
	\quad U =
	\begin{bmatrix}
		\tilde{\Psi}^TX \\
		0
	\end{bmatrix}
\end{equation}
\begin{equation*}
	C = -
	\begin{bmatrix}
		\left(X^TX\right)^{-1}
	\end{bmatrix}
	\quad V = U^Ti.
\end{equation*}

The general setting above is simplified in the rest of this work by assuming:
\begin{itemize}
	\item[--] same domain for all the patients, \ie $\Omega_1 = \ldots = \Omega_m:=
			\Omega$; \item[--] same locations $\bm{p}_{ij}$ for every patient $i$, that is
		$\bm{p}_{ij}=\bm{p}_{kj} \quad \forall (i,k) \in \{ 1 \ldots m \}^2$. This also
		implies that $n_i$, number of observations for patient $i$, is the same for
		every $i$. In the following we will use $n=n_i$. \item[--] same finite element
		basis used across all $m$ domains $\Omega$.
\end{itemize}
These properties allow storing in memory some of the matrices
described above just for one patient rather than for all patients.
