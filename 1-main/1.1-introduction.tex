% !TEX root = ../thesis.tex

\chapter{Introduction}
\label{ch:intro}

\section{Spatial regression with partial differential equation regularization}
This work naturally stems from the studies in the research field called Spatial
Regression with Partial Differential Equation regularization, abbreviated in
the following as SR-PDE. SR-PDE constitutes a family of models which has been,
and still is, under study beginning from \citeyear{sangalli0} in
\citeauthor{sangalli0} \cite{sangalli0}.

SR-PDE effectively combines several branches of mathematics and statistics
modeling. For a thorough analysis of its derivation and the enhancements it
brings to the existing literature see \cite{sangalli1}, I will limit myself to
a brief introduction.\\ These models feature the minimization of a functional
composed of two terms: a classical least-square term, with the goal of
estimating a vector of unknown regression coefficients (in case of presence of
covariates), and a regularization term, estimating an unknown deterministic
spatial field.\\ The spatial field contributes to the functional by integration
over the spatial domain of the square of a differential operator, most commonly
the Laplacian operator. By properties of the Laplace operator, this choice
induces an isotropic and stationary smoothing effect on the unknown spatial
field, meaning equal in every direction and independent on the position. Other
differential operators considered in this work are generic second order
differential operators, that come from the physics of the problem or from some
preexisting knowledge about the phenomenon under study.

The inclusion of partial differential equations in the statistical model brings
many advantages, like the aforementioned ability to include problem specific
knowledge, dealing with boundary conditions, etc., but makes computations more
cumbersome.

\section{Mixed-effects models}
In a situation where observed data posses a natural grouping structure,
mixed-effects models are often utilized. They combine fixed effects, meaning
that a set of covariates have regression coefficients shared among all groups,
and random effects, meaning the remaining set of covariates have regression
coefficients varying along each group.

Given our interest in spatial data analysis, a typical situation where
mixed-models are used is in clinical data, when data coming from several
patients are measured in the same area of the body. As an application we will
see a mixed-effects model applied to human brain data. The data come from a set
of patients who undertook functional Magnetic Resonance Imaging (fMRI) and in
the following we will use the word \textit{patients} to indicate the different
groups of our generic mixed-effects model.

We are therefore ready to present, similarly as in \cite{kim}, the SR-PDE
mixed-effects model.

\section{Generic SR-PDE mixed-effects model}

Consider $m$ patients. To the $i$-th patient, with $i$ varying from 1 to m,
corresponds a unique spatial domain $\Omega_i$, on which we observe $n_i$ data
at different positions $\bm{p}_{ij}$. Index $j$ is therefore varying from $1$
to $n_i$, for given patient $i$.\\The variable of interest z, observed at
$\bm{p}_{ij}$, is modeled as
\begin{equation}
	\label{model}
	z_{ij} = \bm{w}^T_{ij} \bm{\beta} + \bm{v}^T_{ij} \bm{b}_i + f_i(\bm{p_{ij}}) + \epsilon_{ij}.
\end{equation}
The notation utilized represents the following quantities:
\begin{itemize}
	\item $\bm{w}_{ij}$ is the vector of fixed effects covariates for the observation $z_{ij}$;
	\item $\bm{\beta}$ is the vector of regression coefficients for fixed effects;
	\item $\bm{v}_{ij}$ is the vector of random effects covariates for the observation $z_{ij}$;
	\item $\bm{b}_i$ is the vector of regression coefficients of the random
	      effects for patient $i$.
\end{itemize}
Since when we are going to estimate the unknown quantities without
prior knowledge on the field $f$, we are going to assume that the random effect
$\bm{b_i}$ has average $0$:
\begin{equation}
	\sum_{i=1}^{m}{\bm{b_i}}=0.
\end{equation}
Let also $N$ be equal to $\sum_{i=1}^{m}n_i$. For every $i$, $j$
equation \ref{model} constitutes a system of $N$ equations, that is expressed
concisely in algebraic form in the following equations:
\begin{equation}
\bm{z}_i = \bm{W}_i \bm{\beta} + \bm{V}_i \bm{b}_i + \bm{f_i} + \bm{\epsilon_i} \quad i=1 \dots m,
\end{equation}
\begin{equation}
	\bm{z} = \bm{W} \bm{\beta} + \bm{V} \bm{b} + \bm{f_N} + \bm\epsilon.
\end{equation}
\begin{itemize}
	\item $\bm{z}_i \in \R^{n_i}$ is the vector of observed data for patient $i$;
	\item $W_i$ is the matrix whose element $w_{j,k}$
\end{itemize}
