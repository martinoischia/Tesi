% !TEX root = ../thesis.tex

\chapter{Conclusions and future developments} \label{ch:Conclusions} In
previous sections we have shown that the iterative method is a valid
alternative to the more direct computation for the solution. It has been
implemented in the \texttt{fdaPDE} library for the bidimensional case and the
manifold case. Some further extension and generalization has to be carried out,
as well as some optimization in the regarding the layout of the matrices and
their multiplication.

While the focus has been on models sharing the same domain and location points,
future work should include multiple domains with different location points.
The 3D case has still to be implemented, and the only strategy used here for
estimating the value of $lambda$ was computing the model on a grid of values.
The methodology could also be possibly extended to spatio-temporal data.

%The current models have a single smoothing parameter, gamma, for each
%statistical unit, but it is possible to develop models with different
%smoothing parameters for each unit, allowing for consideration of individual
%functional wiggliness. Additionally, if problem-specific information about the
%spatial field is known, such as from physics, mechanics, or chemistry, the
%models can be extended to include different penalty terms with general PDEs Lf
%= u, as demonstrated in [2] and [3]. These extensions can also incorporate
%anisotropy or non-stationarity, as in [4].
%
%The methodology can also be extended to spatio-temporal data, as demonstrated
%in [1] and [5]. However, this extension would significantly increase
%computational costs and may not be feasible from an implementation standpoint.
%Finally, other discretization techniques, such as Non-Uniform Rational
%B-Splines (NURBS), can be adopted, as demonstrated in [16], which explores the
%use of isogeometric analysis with high smoothness.
