% !TEX root = ../thesis.tex

\chapter{Conclusions and future developments} \label{ch:Conclusions}

While the focus has been on models sharing the same domain and location points,
future work should include multiple domains with different location points

The proposed model is not working yet on the fMRI data application but I will
post soon the corrected results, eventually on my GitHub channel \ref{MI}.

%The current models have a single smoothing parameter, gamma, for each
%statistical unit, but it is possible to develop models with different smoothing
%parameters for each unit, allowing for consideration of individual functional
%wiggliness. Additionally, if problem-specific information about the spatial
%field is known, such as from physics, mechanics, or chemistry, the models can
%be extended to include different penalty terms with general PDEs Lf = u, as
%demonstrated in [2] and [3]. These extensions can also incorporate anisotropy
%or non-stationarity, as in [4].
%
%The methodology can also be extended to spatio-temporal data, as demonstrated
%in [1] and [5]. However, this extension would significantly increase
%computational costs and may not be feasible from an implementation standpoint.
%Finally, other discretization techniques, such as Non-Uniform Rational
%B-Splines (NURBS), can be adopted, as demonstrated in [16], which explores the
%use of isogeometric analysis with high smoothness.
