% !TEX root = ../thesis.tex

\chapter{My chapter}
\label{ch:chapter_name}
\section{Matrices}
Assuming $W'_i$ and $V'_i$ for $i=1,\dots,m$ full rank, define the following matrices:\\
\begin{equation*}
    X =    
    \begin{bmatrix}
        W'_1  & V'_1  & 0      & \ldots    & 0      & 0 \\
        W'_2  & 0      & V'_2  & \ldots      & 0      & 0 \\
        \vdots & \vdots & \vdots & \ddots & \vdots & \vdots \\
        W'_{m-1} & 0   & 0      & \ldots      & V'_{m-1} & 0 \\
        W'_m  & 0      & 0      & \ldots      & 0      & V'_m
    \end{bmatrix}
\end{equation*}
with $X\in \R^{N\times(m-1)p+q}$,
$H = X(X^TX)^{-1}X^T$ and $Q = I_N - H$, $\in \R^{N\times N}$, which are the matrixes that project a vector, respectively, onto the subspace spanned by the columns of X and onto its orthogonal complement with respect to $\R^N$.
Notice that despite the matrix $X^TX$ exhibits a pattern where only the first row, the first column and the diagonal are different from 0, the inverse of this type of matrix is usually dense (\textit{might be interesting studying an incomplete factorization of this matrix though}).

The discrete problem leads to the solution of the following linear system of equations:

\begin{equation}
    \label{mono}
    \begin{bmatrix}
        \tilde{\Psi}^TQ\tilde{\Psi} & -\lambda \tilde{R}_1\\
        -\lambda \tilde{R}_1  & -\lambda \tilde{R}_0
    \end{bmatrix}
    \begin{bmatrix}
        \hat{\mathbf{f}}\\
        \hat{\mathbf{g}}
    \end{bmatrix}
    = 
    \begin{bmatrix}
        \tilde{\Psi}^T Q\mathbf{z}\\
        0
    \end{bmatrix}
\end{equation}
We call it monolithic because the number of units might be big and so the number of nodes for each unit. Therefore our aim is to avoid the solution of such high dimension linear system, in favour of splitting it into many systems of lower dimension.
In case the dimensions of the monolithic system are treatable, the Woodbury decomposition formula, described in appendix \ref{sec:wood}, can be used to speed up the computation of the solution for different values of $\lambda$. The decomposition is entirely analogous to the one described in the appendix, with
\begin{equation}
    \label{eq:wootilde}
    E =
    \begin{bmatrix}
        \tilde{\Psi}^T\tilde{\Psi} & -\lambda \tilde{R}_1\\
        -\lambda \tilde{R}_1  & -\lambda \tilde{R}_0
    \end{bmatrix}
\quad
    U = 
    \begin{bmatrix}
        \tilde{\Psi}^TX\\
        0
    \end{bmatrix}
\end{equation}
\begin{equation*}
    C = -
    \begin{bmatrix}
        (X^TX)^{-1}
    \end{bmatrix}
\quad \quad
    V = U^T
\end{equation*}

\section{Iterative methods}

Following the ideas stemmed from the spatio-temporal regression in \citeauthor{pollini} \cite{pollini} and \citeauthor{massardi} \cite{massardi}, we consider an iterative scheme. At each step the scheme computes an approximate solution of the monolithic system by solving a single-unit problem for each statistical unit.
The algorithm stops when one of the following criteria is met:
\begin{itemize}
    \item A maximum number of iterations is reached;
    \item The functional of the estimated solution $J_{\Omega_i, \lambda}(\hat{f}^i)$ has reached stagnation, that is the relative increment $(J_{\Omega_i, \lambda}(\hat{f}^i) - J_{\Omega_i, \lambda}(\hat{f}^{i-1})) / J_{\Omega_i, \lambda}(\hat{f}^i)$ is below a certain threshold (in the code $10^{-4}$ was used);
    \item The estimated solution has reached stagnation, that is the relative increment $\norm{\hat{f}^i-\hat{f}^{i-1}}/\norm{\hat{f}^i}$ is below a certain threshold (in the code $10^{-8}$ was used).
\end{itemize}
The details of a first possible implementation are described in the following section.


\subsection{The block diagonal approach}
\label[subsection]{iter}

Looking at the monolithic equation \ref{mono}, the question of how to formulate an approximation of the term $\tilde{\Psi}^TQ\tilde{\Psi}$ naturally arises. 
In particular, a suitable block approximation allows to make the estimated field $\hat{f}_i$ of statistical unit $i$ independent from the observations in the other units, for every unit $i$.

To this purpose, the following approximation is first considered:

\begin{equation}
    \label{blockdia}
    \tilde{\Psi}^TQ\tilde{\Psi} =
    \begin{bmatrix}
        \Psi^TQ_1\Psi & 0& \dots & 0\\
        0 & \Psi^TQ_2\Psi & \ddots & \vdots\\
        \vdots& \ddots& \ddots & 0 \\
        0 & \dots & 0 & \Psi^TQ_m\Psi\\
    \end{bmatrix}    
\end{equation}
Here $Q_i$ indicates the $i$-th diagonal block of $Q$ of dimension $n \times n$. By definition of $Q$, it is equal to $I_n-X_i(X^TX)^{-1}X_i^T$, with $X_i=X((i-1)m+1 : im\,, \;:\; )$, where the typical notation of the MATLAB language has been used to express a suitable submatrix.\\
Another possible way to express the blocks $\Psi^TQ_i\Psi$ of matrix \ref{blockdia} is through Woodbury decomposition (appendix \ref{sec:wood}) with a use analogous to the one in \ref{eq:wootilde}. In fact, $\Psi^TQ_i\Psi$ is the $i$-th block diagonal of matrix $\tilde{\Psi}^TQ\tilde{\Psi}$, which in turn can be expressed as $\tilde{\Psi}^T\tilde{\Psi}-\tilde{\Psi}^TH\tilde{\Psi}$. Thus, defining $U = \tilde{\Psi}^TX$, $C = -(X^TX)^{-1}$,$V =U^T$, matrix $\Psi^TQ_i\Psi$ can be expressed as $\Psi^T\Psi +  U_iCV_i $, where $U_i = U((i-1)m+1 : im\,, \;:\;)$, $V_i = U_i^T $.

\subsection{Initialization}
The inizialization consists in finding a good guess $(\hat{\mathbf{f}}^0,\hat{\mathbf{g}}^0)$ to start the algorithm from; for this purpose, the following $m$ problems are solved: for $i = 1, \dots, m$ solve
\begin{equation}
        \begin{bmatrix}
            \Psi^TQ_i\Psi & -\lambda R_1\\
            -\lambda R_1  & -\lambda R_0
        \end{bmatrix}
        \begin{bmatrix}
            \hat{\mathbf{f}}_i^0\\
            \hat{\mathbf{g}}_i^0
        \end{bmatrix}
        =
        \begin{bmatrix}
            \mathbf{u}_i\\
            0
        \end{bmatrix}
\end{equation}
where $\mathbf{u}_i$ is the vector whose components are the first $n$ components of $\tilde{\Psi}^T Q\mathbf{z}$ starting from the $n(i-1) +1$-th component.

\subsection{Iterations}
The idea behind the iterative scheme is, having a guess of a solution $(\hat{\mathbf{f}}^{k-1},\hat{\mathbf{g}}^{k-1})$, to compute a new guess $(\hat{\mathbf{f}}^k,\hat{\mathbf{g}}^k)$ by replacing the mutual interaction of the system variables corresponding to different units with their contibution given by their previous value.\\
For example, the first $n$ equations of the monolithic system \ref{mono} read
\begin{equation}
    \begin{bmatrix}
        \Psi^TQ_{1,1}\Psi & \Psi^TQ_{1,2}\Psi &\dots &\Psi^TQ_{1,m}\Psi & -\lambda R_1\\
    \end{bmatrix}
    \begin{bmatrix}
        \hat{\mathbf{f}}_1\\
        \hat{\mathbf{f}}_2\\
        \vdots\\
        \hat{\mathbf{f}}_m\\
        \hat{\mathbf{g}}_1    
    \end{bmatrix}
    =
    \begin{bmatrix}
        \mathbf{u}_1
    \end{bmatrix}
\end{equation}
where $Q_{i,j}$ indicates block of row $i$ and column $j$ of matrix $Q$.
Substituting $\hat{\mathbf{f}}_j$ for $j=1,\dots,m$ with $\hat{\mathbf{f}}_j^{k-1}$ and generalizing, the iterative scheme reads: for $i = 1, \dots, m$ solve
\begin{equation}
    \begin{bmatrix}
        \Psi^TQ_i\Psi & -\lambda R_1\\
        -\lambda R_1  & -\lambda R_0
    \end{bmatrix}
    \begin{bmatrix}
        \hat{\mathbf{f}}_i^k\\
        \hat{\mathbf{g}}_i^k
    \end{bmatrix}
    =
    \begin{bmatrix}
        \mathbf{r}_i\\
        0
    \end{bmatrix}
\end{equation}
where
\begin{equation}
     \mathbf{r}_i =\mathbf{u}_i -\sum_{\substack{j=1\\ j\neq i}}^m \Psi^TQ_{i,j} \Psi \hat{\mathbf{f}}_j^{k-1}
\end{equation}

