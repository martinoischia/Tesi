% !TEX root = ../thesis.tex

\chapter{My chapter}
\label{ch:chapter_name}
\section{Functional}
...by minimizing the following functional:
\begin{equation}
    \label{functional}
    \begin{split}  
        J_{\Omega_i, \lambda} \left(\bm{\beta'}, \mathbf{b'_1}, \dots, \mathbf{b'_m}, f_1, \dots, f_m \right) = \\ \sum_{i = 1}^m \left( \sum_{j=1}^{n_i} \left( z_{ij}-\mathbf{w'_{ij}}^T \bm{\beta'} - \mathbf{v'_{ij}}^T \mathbf{b'_i} - f_i(\mathbf{p_{ij}}) \right)^2 + \lambda \int_{\Omega_i} \Delta f_i \left(\bm{p}\right)^2 d\Omega_i\right)
    \end{split}
    \end{equation}
\section{Matrices}
Assuming $W'_i$ and $V'_i$ for $i=1,\dots,m$ full rank, define the following matrices:
\begin{equation}
    X =    
    \begin{bmatrix}
        W'_1  & V'_1  & 0      & \ldots    & 0      & 0 \\
        W'_2  & 0      & V'_2  & \ldots      & 0      & 0 \\
        \vdots & \vdots & \vdots & \ddots & \vdots & \vdots \\
        W'_{m-1} & 0   & 0      & \ldots      & V'_{m-1} & 0 \\
        W'_m  & 0      & 0      & \ldots      & 0      & V'_m
    \end{bmatrix}
\end{equation}
with $X\in \R^{N\times(m-1)p+q}$,
$H = X\left(X^TX\right)^{-1}X^T$ and $Q = I_N - H$, $\in \R^{N\times N}$, which are the matrices that project a vector, respectively, onto the subspace spanned by the columns of X and onto its orthogonal complement with respect to $\R^N$.
Notice that despite the matrix $X^TX$ exhibits a pattern where only the first row, the first column and the diagonal are different from 0, the inverse of this type of matrix is usually dense.

Define also the vector of coefficients $\bm{\nu} = (\bm{\beta}^\prime, \bm{b}_1^\prime, \dots, \bm{b}_m^\prime)$.\\
Given these definitions, we can write the mixed-effect model in implementation version with this compact formula
\begin{equation}
    \label{modelX}
    \bm{z} = X \bm{\nu} + \bm{f}_N + \bm{\epsilon}
\end{equation}

The discrete problem leads to the solution of the following linear system of equations:

\begin{equation}
    \label{mono}
    \begin{bmatrix}
        \tilde{\Psi}^TQ\tilde{\Psi} & -\lambda \tilde{R}_1^T\\
        -\lambda \tilde{R}_1  & -\lambda \tilde{R}_0
    \end{bmatrix}
    \begin{bmatrix}
        \hat{\mathbf{f}}\\
        \hat{\mathbf{g}}
    \end{bmatrix}
    = 
    \begin{bmatrix}
        \tilde{\Psi}^T Q\mathbf{z}\\
        0
    \end{bmatrix}
\end{equation}
We call it monolithic because the number of units might be big and so the number of nodes for each unit. Therefore, our aim is to avoid the solution of such high dimension linear system, in favor of splitting it into many systems of lower dimension.
In case the dimensions of the monolithic system are treatable, the Woodbury decomposition formula, described in appendix \ref{sec:wood}, can be used to speed up the computation of the solution for different values of $\lambda$. The decomposition is entirely analogous to the one described in the appendix, with
\begin{equation}
    \label{eq:wootilde}
    E =
    \begin{bmatrix}
        \tilde{\Psi}^T\tilde{\Psi} & -\lambda \tilde{R}_1^T\\
        -\lambda \tilde{R}_1  & -\lambda \tilde{R}_0
    \end{bmatrix}
\quad
    U = 
    \begin{bmatrix}
        \tilde{\Psi}^TX\\
        0
    \end{bmatrix}
\end{equation}
\begin{equation*}
    C = -
    \begin{bmatrix}
        \left(X^TX\right)^{-1}
    \end{bmatrix}
\quad \quad
    V = U^T
\end{equation*}

Once the unknown spatial field $\hat{\bm{f}}$ has been obtained, the vector of unknown coefficients $\hat{\bm{\nu}} = (\hat{\bm{\beta}}^\prime, \hat{\bm{b}}_1^\prime, \dots, \hat{\bm{b}}_m^\prime)$ can be computed solving a least-squares problem: the corresponding normal equations are written as
    
\begin{equation}
    \label{nu}
    \hat{\bm{\nu}} = \left(X^TX\right)^{-1}X^T\left(\bm{z} - \hat{\bm{f}}_N\right)
\end{equation}
where I recall that $\hat{\bm{f}}_N = \tilde{\Psi} \hat{\bm{f}}$.

\section{Iterative methods}

Following the ideas stemmed from the spatio-temporal regression in \citeauthor{pollini} \cite{pollini} and \citeauthor{massardi} \cite{massardi}, we consider an iterative scheme. At each step $i$ the scheme computes an approximate solution $(\hat{\mathbf{f}}^{i},\hat{\mathbf{g}}^{i})$ of the monolithic system by solving a single-unit problem for each statistical unit.
The algorithm stops with two possible criteria:
\begin{itemize}
    \item A maximum number of iterations is reached;
    \item The following two conditions are true. \\
    The first one is that the functional \ref{functional} evaluated in the estimated solution $J^i = J_{\Omega_i, \lambda}\left( \hat{\bm{\beta}}^{\prime i}, \hat{\bm{b}}^{\prime i}_1, \dots, \hat{\bm{b}}^{\prime i}_m, \hat{f}_1^i, \dots, \hat{f}_m^i \right)$ has reached stagnation, that is the relative increment $\left( J^i - J^{i-1}\right) / J^i$ is below a certain threshold (in the code such threshold is an input parameter, it was set to $10^{-8}$). The term $ \int_{\Omega_i} \Delta f_i \left(\bm{p}\right)^2 d\Omega_i $ of the functional \ref{functional} is computed by exploiting the expansion into the finite element basis functions, leading to $\hat{\bm{g}}_i^T R_0 \hat{\bm{g}}_i$, where $\hat{\bm{g}}_i $ is the sub-vector of $\hat{\bm{g}}$ corresponding to unit $i$.\\
    The second one is that the estimated solution is very close to the exact solution of the system \ref{mono}, condition that is verified by checking that the residual, normalized by the Euclidean norm of the right-hand side, is below a certain threshold (this is another input parameter, $10^{-8}$ was used).
\end{itemize}
The details of a first possible implementation are described in the following section.

\subsection{The block diagonal approach}
\label[subsection]{iter}

Looking at the monolithic equation \ref{mono}, the question of how to formulate an approximation of the term $\tilde{\Psi}^TQ\tilde{\Psi}$ naturally arises. 
In particular, a suitable block approximation allows to make the estimated field $\hat{f}_i$ of statistical unit $i$ independent of the observations in the other units, for every unit $i$ ($i = 1 \dots m$).

To this purpose, the following approximation is first considered:

\begin{equation}
    \label{blockdia}
    \tilde{\Psi}^TQ\tilde{\Psi}\simeq \Gamma := 
    \begin{bmatrix}
        \Psi^TQ_1\Psi & 0& \dots & 0\\
        0 & \Psi^TQ_2\Psi & \ddots & \vdots\\
        \vdots& \ddots& \ddots & 0 \\
        0 & \dots & 0 & \Psi^TQ_m\Psi\\
    \end{bmatrix}    
\end{equation}
Here $Q_i$ indicates the $i$-th diagonal block of $Q$ of dimension $n \times n$. By definition of $Q$, it is equal to $I_n-X_i\left(X^TX\right)^{-1}X_i^T$, with $X_i=X\left(\left(i-1\right)n+1 : in\,, \;:\; \right)$, where the typical notation of the MATLAB language has been used to express a suitable submatrix.\\
Another possible way to express the blocks $\Psi^TQ_i\Psi$ of matrix \ref{blockdia} is through Woodbury decomposition (appendix \ref{sec:wood}) with a use analogous to the one in \ref{eq:wootilde}.\\
 To show it, we notice that $\Psi^TQ_i\Psi$ is the $i$-th diagonal block of matrix $\tilde{\Psi}^TQ\tilde{\Psi}$.
In fact $\tilde{\Psi}^TQ\tilde{\Psi}$ can be expressed as $\tilde{\Psi}^T\tilde{\Psi}-\tilde{\Psi}^TH\tilde{\Psi}$. Thus, defining $U = \tilde{\Psi}^TX$, $C = -\left(X^TX\right)^{-1}$,$V =U^T$, its $i$-th diagonal block is equal to $\Psi^T\Psi +  U_iCV_i $, where
\begin{equation}
    \begin{split}
        U_i = U\left(\left(i-1\right)N_\mathcal{T}+1 : iN_\mathcal{T}\,, \;:\;\right) = \tilde{\Psi}^T_iX =\\
        \begin{bmatrix}
            0 & \dots & 0 & \Psi^T & 0 & \dots & 0
        \end{bmatrix}
        \begin{bmatrix}
            X_1\\
            \vdots\\
            X_i\\
            \vdots\\
            X_m
        \end{bmatrix}
        = \Psi^T X_i
    \end{split}
\end{equation}
\begin{equation}
    V_i = U_i^T = X^T_i\Psi
\end{equation}
In such a way we have
\begin{equation}
    \Psi^T\Psi +  U_iCV_i = \Psi^T \left( I_n-X_i\left(X^TX\right)^{-1}X_i^T \right) \Psi = \Psi^T Q_i \Psi
\end{equation}

\subsection{Initialization}
The inizialization consists in finding a good guess $(\hat{\mathbf{f}}^0,\hat{\mathbf{g}}^0)$ to start the algorithm from; for this purpose, the following $m$ problems are solved: for $i = 1, \dots, m$ solve
\begin{equation}
        \begin{bmatrix}
            \Psi^TQ_i\Psi & -\lambda R_1^T\\
            -\lambda R_1  & -\lambda R_0
        \end{bmatrix}
        \begin{bmatrix}
            \hat{\mathbf{f}}_i^0\\
            \hat{\mathbf{g}}_i^0
        \end{bmatrix}
        =
        \begin{bmatrix}
            \mathbf{u}_i\\
            0
        \end{bmatrix}
\end{equation}
where $\mathbf{u}_i$ is the vector whose components are the first $n$ components of $\tilde{\Psi}^T Q\mathbf{z}$ starting from the $n(i-1) +1$-th component.

\subsection{Iterations}
The idea behind the iterative scheme is, having a guess of a solution $(\hat{\mathbf{f}}^{k-1},\hat{\mathbf{g}}^{k-1})$, to compute a new guess $(\hat{\mathbf{f}}^k,\hat{\mathbf{g}}^k)$ by replacing the mutual interaction of the system variables corresponding to different units with their contibution given by their previous value.\\
For example, the first $n$ equations of the monolithic system \ref{mono} read
\begin{equation}
    \begin{bmatrix}
        \Psi^TQ_{1,1}\Psi & \Psi^TQ_{1,2}\Psi &\dots &\Psi^TQ_{1,m}\Psi & -\lambda R_1^T\\
    \end{bmatrix}
    \begin{bmatrix}
        \hat{\mathbf{f}}_1\\
        \hat{\mathbf{f}}_2\\
        \vdots\\
        \hat{\mathbf{f}}_m\\
        \hat{\mathbf{g}}_1    
    \end{bmatrix}
    =
    \begin{bmatrix}
        \mathbf{u}_1
    \end{bmatrix}
\end{equation}
where $Q_{i,j}$ indicates block of row $i$ and column $j$ of matrix $Q$, which also coincides with $-H_{i,j}$ for $i \neq j$.
Substituting $\hat{\mathbf{f}}_j$ for $j=1,\dots,m$ with $\hat{\mathbf{f}}_j^{k-1}$, taking it to the right-hand side and generalizing, the iterative scheme reads: for $i = 1, \dots, m$ solve
\begin{equation}
    \begin{bmatrix}
        \Psi^TQ_i\Psi & -\lambda R_1^T\\
        -\lambda R_1  & -\lambda R_0
    \end{bmatrix}
    \begin{bmatrix}
        \hat{\mathbf{f}}_i^k\\
        \hat{\mathbf{g}}_i^k
    \end{bmatrix}
    =
    \begin{bmatrix}
        \mathbf{r}_i\\
        0
    \end{bmatrix}
\end{equation}
where
\begin{equation}
     \mathbf{r}_i =\mathbf{u}_i -\sum_{\substack{j=1\\ j\neq i}}^m \Psi^TQ_{i,j} \Psi \hat{\mathbf{f}}_j^{k-1}
\end{equation}

An estimate of $\bm{\nu}$, $\hat{\bm{\nu}}^i = (\hat{\bm{\beta}}^{\prime i}, \hat{\bm{b}}_1^{\prime i}, \dots, \hat{\bm{b}}_m^{\prime i})$ has to be computed at each iteration for the estimation of the functional \ref{functional}, according to equation \ref{nu}.

\subsection{The iterative method as a preconditioned Richardson method}
Given a generic preconditioning matrix $P$, calling $\bm{r}_k$ the residual of the linear system $A \bm{x} = \bm{b}$ (that is the vector $A \bm{x}_k - \bm{b}$ )at step $k$,
Richardson method (\textit{cf.}, for example, \cite{Quarteroni})consists in solving (or rather trying to) the linear system iterating the following steps:
\begin{enumerate}
    \item Solve $P \bm{z}_k = \bm{r}_k$
    \item Compute the acceleration parameter $\alpha_k$ (for simplicity we use $\alpha_k = 1$)
    \item Update the solution $ \bm{x}_{k+1} = \bm{x}_{k} - \alpha_k \bm{z}_k $
    \item Update the residual $\bm{r}_{k+1} = \bm{r}_{k} - \alpha_k A \bm{z}_k$
\end{enumerate}

The iterative scheme described in the previous section is a Richardson scheme with the following preconditioning matrix:
\begin{equation}
    \label{precond}
    P=
    \begin{bmatrix}
        \Gamma & -\lambda \tilde{R}_1^T\\
        -\lambda \tilde{R}_1  & -\lambda \tilde{R}_0
    \end{bmatrix}
\end{equation}
Solving an awfully hard linear system involving this matrix is indeed like solving $m$ linear systems of dimensions $2n\times 2n$.
\section{Generalized cross validation}
\underline{As we mentioned before}, the system is solved multiple times on a grid of $\lambda s$. The best $\lambda$ is then chosen according
to a generalized cross-validation criterion. In particular, as in \cite{sangalli1}, the minimum of the generalized cross-validation (GCV) parameter is used as a model selection criterion. The generalization of this parameter to the mixed-effect model reads as follows:
\begin{equation}
    \label{GCV}
    GCV(\lambda) = \frac{1}{N\left(1-\left(q-p+mp+tr\left(S\right)\right)/N\right)^2}\norm{\bm{z} - \hat{\bm{z}}}^2
\end{equation}
\underline{where for $q$, $m$, $p$ we follow the notation of section ?}, and S indicates the smoothing matrix, which is the matrix that maps the observations vector $\bm{z}$ into the estimated spatial field $\hat{\bm{f}}$ evaluated at the location of the observations ($\,\hat{\bm{f}}_N = \tilde{\Psi}\hat{\bm{f}} = S z\,$). Its value is
\begin{equation}
        S = \tilde{\Psi} \left(\tilde{\Psi}^TQ\tilde{\Psi} + \lambda \tilde{R}_1^T
        \tilde{R}_0^{-1} \tilde{R}_1\right)^{-1}\tilde{\Psi}^TQ
\end{equation}
and it stems as a Schur complement for system \ref{mono} with respect to the estimated field $\hat{\bm{f}}$.\\
The computation of the GCV parameter \ref{GCV} involves the expensive computation of the trace of the smoothing matrix $S$.
Since for any two matrices $A$ and $B$ such that the product $AB$ is defined, it holds that $tr\left(AB\right) = tr\left(BA\right)$, the trace of $S$ can be computed as the trace of
\begin{equation}
    \tilde{S} = \left(\tilde{\Psi}^TQ\tilde{\Psi} + \lambda \tilde{R}_1^T
    \tilde{R}_0^{-1} \tilde{R}_1\right)^{-1}\tilde{\Psi}^TQ \tilde{\Psi}
\end{equation}

