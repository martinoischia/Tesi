% !TEX root = ../thesis.tex
\chapter{Applications} \label{ch:Applications}

Functional Magnetic Resonance Imaging (fMRI) is an imaging technique developed
to detect regional, time-varying changes in brain metabolism, which can arise
from task-induced cognitive state changes or unregulated processes in the
resting brain. Since the 90s, fMRI has become widely used in the context of
cognitive neurosciences, psychiatry and psychology and a large number of studies are available on it. The
popularity of fMRI can be attributed to its widespread availability,
non-invasive nature, relatively low cost, and good spatial resolution.

In this study, we want to prosecute the work started in \citeauthor
\cite{kim}. In the appendix of her thesis the interested reader
can find the description on how to access all the databases we used, and also the patients that she utilized in her study.
For the sake of the mixed-effects model, the statistical units are several patients on which
fMRI observations were taken as a time-series.

As described in \citeauthor{kim}, we first process the large amount of time-series data (1200
rows per brain location), and we move to analyze the so-called Functional
Connectivity (FC) maps. FC maps are constructed by computing pairwise
correlations between the fMRI time-series of all brain locations and the mean
time-series of a Region of Interest (ROI) on the cortical surface.
Further, a Fisher transformation is applied on the obtained variable. In this way,
we can shed some light in identifyng the areas of the
brain that are functionally connected to the ROI. 

As done before in similar studies, we consider precuneus as the Region Of Interest.
We use the available data of the cortical thickness of
the cerebral cortex as a covariate for our mixed-effects model.
The cortex is in fact known to stimulate neural activity.

For this type of application it is possible to consider the thin cerebral cortex as a 2D surface embedded in a 3D space.
Despite the unique geometrical properties of each individual brain, we use a common
standard surface provided by the Conte69 brain atlas, as well as a common mesh.
Moreover, as it was done in \cite{kim} we limit ourselves to the left hemisphere.
The Conte69 brain atlas contains a large number of locations (32492) in the
brain. This locations correspond with the locations on which we observe the fMRI
data.

\citeauthor{kim}, not having at her disposal a version of the library that exploits
the convenience of the use of sparse matrices, as well as the iterative method that
greatly reduces the amount of RAM needed, had to perform some simplifications.
She used two different meshes, the original one with 32492 nodes
and a simplified one with 10000 nodes. The latter allowed her to study at same time
more patients while keeping the system manageable.
She considered $12$ patients for the full mesh study and $30$ for
the simplified mesh study, but several hundreds are available.
In my case, I moved to $60$ patients, as each patient data occupy around
one GB of storage.

The mesh strategy that she adopted was well studied and took into account various criteria, such as
preserving the topology and shape of the original mesh, but the results were not
validated.

It is worth noting that these maps use the same scale for all 12
patients. The region of interest, precuneus, is situated in the upper left area
of the brain, as illustrated in Figure 5.4. Given that the connectivity of this
region with itself should be high, this area is colored in red in each
estimated function. Although patients share the same red pattern in the
precuneus, they exhibit different colors in distinct regions on Figure 5.6.
While some patients exhibit significant contrast in certain areas of the brain,
others lack distinctive characteristics. A few patients possess a similar
pattern to that seen in Fig. The function estimates shown in Fig
and exhibit noticeable variations in color between different patients,
indicating that there may be significant differences in brain connectivity
patterns across individuals.
