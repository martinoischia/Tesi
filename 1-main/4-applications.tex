% !TEX root = ../thesis.tex
\chapter{Applications} \label{ch:Applications}

Functional Magnetic Resonance Imaging (fMRI) is an imaging technique developed
to detect regional, time-varying changes in brain metabolism, which can arise
from task-induced cognitive state changes or unregulated processes in the
resting brain. Since the 90s, fMRI has become widely used in the context of
cognitive neurosciences, psychiatry and psychology and a large number of
studies are available on it. The popularity of fMRI can be attributed to its
widespread availability, non-invasive nature, relatively low cost, and good
spatial resolution.

In this study, we want to prosecute the work started in \cite{kim}. In the
appendix of her thesis the interested reader can find the description on how to
access all the databases we used, and also the patients that she utilized in
her study.  For the sake of the mixed-effects model, the statistical units are
several patients on which fMRI observations were taken as a time-series.

As described by \citeauthor{kim}, we first process the large amount of
time-series data (1200 rows per brain location), and we move to analyze the
so-called Functional Connectivity (FC) maps. FC maps are constructed by
computing pairwise correlations between the fMRI time-series of all brain
locations and the mean time-series of a Region of Interest (ROI) on the
cortical surface.  Further, a Fisher transformation is applied on the obtained
variable. In this way, we can shed some light in identifyng the areas of the
brain that are functionally connected to the ROI. 

As done before in similar studies, we consider precuneus as the Region Of
Interest.  We use the available data of the cortical thickness of the cerebral
cortex as a covariate for our mixed-effects model.  The cortex is in fact known
to stimulate neural activity.

For this type of application it is possible to consider the thin cerebral
cortex as a 2D surface embedded in a 3D space.  Despite the unique geometrical
properties of each individual brain, we use a common standard surface provided
by the Conte69 brain atlas, as well as a common mesh.  Moreover, as it was done
in \cite{kim} we limit ourselves to the left hemisphere.  The Conte69 brain
atlas contains a large number of locations (32492) in the brain. This locations
correspond with the locations on which we observe the fMRI data.

\citeauthor{kim}, not having at her disposal a version of the library that
exploits the convenience of the use of sparse matrices, as well as the
iterative method that greatly reduces the amount of RAM needed, had to perform
some simplifications.  She used two different meshes, the original one with
32492 nodes and a simplified one with 10000 nodes. The latter allowed her to
study at same time more patients while keeping the system manageable.  She
considered $12$ patients for the full mesh study and $30$ for the simplified
mesh study, but several hundreds are available.  In my case, I moved to $60$
patients, as my small hard disk was not allowing more.

The mesh strategy that she adopted was well studied and took into account
various criteria, such as preserving the topology and shape of the original
mesh, but the results were not validated.

The iterative method successfully run on a grid of $10$ $\lambda$s, using again
60 as the number of stochastic realizations to estimate the degrees of freedom.
Again the method converged in less than 6 iterations, taking less than two
hours to run, and the parameters estimated are in line with those of
\cite{kim}.
