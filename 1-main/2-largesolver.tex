% !TEX root = ../thesis.tex

\chapter{Solving large linear systems for the mixed-effects model}
\label{ch:chapter_name}
\section{Iterative methods}

Following the ideas stemmed from the spatio-temporal regression in
\citeauthor{pollini} \cite{pollini} and \citeauthor{massardi} \cite{massardi},
we consider an iterative scheme. At each step $k$ the scheme computes an
approximate solution $(\hat{\bm{f}}^{k},\hat{\bm{g}}^{k})$ of the monolithic
system by solving a single-unit problem for each statistical unit. After
initializing the needed quantities, we execute iterations until we satisfy a
stopping criterion.\\ Denote with $J^k$ the functional $J_{\Omega_i, \lambda}$
in \ref{functional}, evaluated in the estimated solution at the $k$-th
iteration:
\begin{equation}
	J^k = J_{\Omega_i, \lambda}\left(
	\hat{\bm{\beta}}^{\prime k}, \hat{\bm{b}}^{\prime k}_1, \dots,
	\hat{\bm{b}}^{\prime k}_m, \hat{f}_1^k, \dots, \hat{f}_m^k \right).
\end{equation}
The algorithm stops with two possible criteria:
\begin{itemize}
	\item[--]A maximum number of iterations is reached;
	\item[--] The following two
		conditions hold:
		\begin{enumerate}
			\item[a)] $J_k$ has reached stagnation, that is the relative increment $\left( J^k - J^{k-1}\right) / J^k$
				is below a certain threshold. In the code such threshold is an input parameter,
				it was set to $10^{-6}$. The term $ \int_{\Omega_i} \Delta f_i
					\left(\bm{p}\right)^2 d\Omega_i $ of the functional \ref{functional} is easily
				computed by exploiting the expansion into the finite element basis functions:
				it is equal to $\hat{\bm{g}}_i^T R_0 \hat{\bm{g}}_i$, where $\hat{\bm{g}}_i $
				is the sub-vector of $\hat{\bm{g}}$ corresponding to unit $i$.
			\item[b)] The estimated solution is close to the exact solution of system
				\ref{mono}. This condition is verified by checking that the residual,
				normalized by the Euclidean norm of the right-hand side, is below a certain
				threshold (this is another input parameter, $10^{-6}$ was used).
		\end{enumerate}
\end{itemize}
The details of the method that was implemented are described in the
following sections.

\subsection{A block diagonal approximation}
\label[subsection]{iter}

Looking at the monolithic equation \ref{mono}, the question of how to formulate
an approximation of the term $\tilde{\Psi}^TQ\tilde{\Psi}$ naturally arises. In
particular, a suitable block approximation allows to make the estimated field
$\hat{f}_i$ of statistical unit $i$ independent of the observations in the
other units, for every unit $i$ ($i = 1 \dots m$). To this purpose, the
following approximation is considered:

\begin{equation}
	\label{blockdia}
	\tilde{\Psi}^TQ\tilde{\Psi}\simeq \Gamma :=
	\begin{bmatrix}
		\Psi^T_1 Q_1\Psi_1 & 0      & 0                 \\
		0                  & \ddots & 0                 \\
		0                  & 0      & \Psi^T_mQ_m\Psi_m \\
	\end{bmatrix}
	,
\end{equation}
where $Q_i$ indicates the $i$-th diagonal block of $Q$, of dimensions
$n_i \times n_i$. By definition of $Q$, this is equal to
$I_{n_i}-X_i\left(X^TX\right)^{-1}X_i^T$, with $X_i$ being the submatrix of $X$
with coefficients $x_{jk}$, with $j$ belonging to $\{\sum_{l=0}^{i-1}{n_l}+1,
	\dots,\sum_{l=0}^{i}{n_l}\}$, $k$ in $\{1,\dots,q-p+mp\}$.\\ This choice
naturally stems from the idea behind the iterative method as we will see in
section \ref{iterburocratico}, but also $\Psi^TQ_i\Psi$ is the $i$-th diagonal
block of matrix $\tilde{\Psi}^TQ\tilde{\Psi}$. In fact, this is equal to
\begin{equation}
	\tilde{\Psi}^T_iQ\tilde{\Psi}_i
\end{equation}
where here we have indicated with $\tilde{\Psi}^T_i$,
$\tilde{\Psi}_i$ the $i$-th block row submatrix, \cf \ref{Psitilde} with the
definition of $\tilde{\Psi}$. By denoting with $Q_{i,j}$ the submatrix of $Q$
naturally arising from the subdivision per statistical unit (recall that $Q \in
	N \times N$ with $N=\sum_{i=1}^{m}n_i$) and substituting we get
\begin{equation}
	\begin{split}
		\tilde{\Psi}^T_iQ\tilde{\Psi}_i=
		\begin{bmatrix}
			0 & \dots & 0 & \Psi_i^T & 0 & \dots & 0
		\end{bmatrix}
		Q
		\begin{bmatrix}
			0      \\
			\vdots \\
			0      \\
			\Psi_i \\
			0      \\
			\vdots \\
			0
		\end{bmatrix}
		=\\
		\begin{bmatrix}
			\Psi^T_i Q_{i,1} & \dots \Psi^T_i Q_{i,m}
		\end{bmatrix}
		\begin{bmatrix}
			0      \\
			\vdots \\
			0      \\
			\Psi_i \\
			0      \\
			\vdots \\
			0
		\end{bmatrix}
		= \Psi_i^T Q_i \Psi_i.
	\end{split}
\end{equation}

\subsection{Initialization}
The initialization consists in finding a good guess
$(\hat{\bm{f}}^0,\hat{\bm{g}}^0)$ to start the algorithm from; for this
purpose, the following $m$ problems are solved: for $i = 1, \dots, m$ solve
\begin{equation}
	\begin{bmatrix}
		\Psi^T_i Q_i \Psi_i & -\lambda R_1^T \\
		-\lambda R_1        & -\lambda R_0
	\end{bmatrix}
	\begin{bmatrix}
		\hat{\bm{f}}_i^0 \\
		\hat{\bm{g}}_i^0
	\end{bmatrix}
	=
	\begin{bmatrix}
		\bm{u}_i \\
		0
	\end{bmatrix}
	,
\end{equation}
where $\bm{u}_i$ is the subvector $\left( \tilde{\Psi}^T Q\bm{z}
	\right)_k$ such that $k=\sum_{l=1}^{l=i-1} + 1, \dots,\sum_{l=1}^{l=i}$.

\subsection{Iterations}
\label{iterburocratico}
The idea behind the iterative scheme is, having a guess of a solution
$(\hat{\bm{f}}^{k-1},\hat{\bm{g}}^{k-1})$, to compute a new guess
$(\hat{\bm{f}}^k,\hat{\bm{g}}^k)$ by replacing inside the monolithic system
\ref{mono} the unknown coefficients relative to all but one unit, with their
value computed at previous step, for every unit.

For example, the first $n_i$ equations of the monolithic system \ref{mono} read
\begin{equation}
	\begin{bmatrix}
		\Psi_1^TQ_{1,1}\Psi_1 & \Psi_1^TQ_{1,2}\Psi_2 & \dots & \Psi_1^TQ_{1,m}\Psi_m & -\lambda R_1^T \\
	\end{bmatrix}
	\begin{bmatrix}
		\hat{\bm{f}}_1 \\
		\hat{\bm{f}}_2 \\
		\vdots         \\
		\hat{\bm{f}}_m \\
		\hat{\bm{g}}_1
	\end{bmatrix}
	=
	\begin{bmatrix}
		\bm{u}_1
	\end{bmatrix}
	,
\end{equation}
where $Q_{i,j}$ indicates block of row $i$ and column $j$ of matrix
$Q$, as seen in section \ref{iter}. Substituting $\hat{\bm{f}}_j$ for $j \neq
	1$ with $\hat{\bm{f}}_j^{k-1}$, taking previous step values to the right-hand
side and generalizing, the iterative scheme reads: for $i = 1, \dots, m$ solve
\begin{equation}
	\begin{bmatrix}
		\Psi_i^TQ_i\Psi_i & -\lambda R_1^T \\
		-\lambda R_1      & -\lambda R_0
	\end{bmatrix}
	\begin{bmatrix}
		\hat{\bm{f}}_i^k \\
		\hat{\bm{g}}_i^k
	\end{bmatrix}
	=
	\begin{bmatrix}
		\bm{r}_i \\
		0
	\end{bmatrix}
\end{equation}
where
\begin{equation}
	\bm{r}_i =\bm{u}_i -\sum_{\substack{j=1\\ j\neq i}}^m \Psi^T_iQ_{i,j} \Psi_j \hat{\bm{f}}_j^{k-1}
\end{equation}

An estimate of $\bm{\nu}$, $\hat{\bm{\nu}}^i = (\hat{\bm{\beta}}^{\prime i},
	\hat{\bm{b}}_1^{\prime i}, \dots, \hat{\bm{b}}_m^{\prime i})$ has to be
computed at each iteration for the estimation of the functional
\ref{functional}, according to equation \ref{nu}.

\subsection{The iterative method as a preconditioned Richardson method}
Given a generic preconditioning matrix $P$, defining $\bm{r}_k=A \bm{x}_k -
	\bm{b}$ the residual of the linear system $A \bm{x} = \bm{b}$ at step $k$,
Richardson method (\textit{cf.}, for example, \cite{Quarteroni}) consists in
solving (or rather trying to) the linear system iterating the following steps:
\begin{enumerate}
	\item Solve $P \bm{z}_k = \bm{r}_k$
	\item Compute the acceleration parameter $\alpha_k$ (for simplicity we use $\alpha_k = 1$)
	\item Update the solution $ \bm{x}_{k+1} = \bm{x}_{k} - \alpha_k \bm{z}_k $
	\item Update the residual $\bm{r}_{k+1} = \bm{r}_{k} - \alpha_k A \bm{z}_k$
\end{enumerate}

The iterative scheme described in the previous section is a Richardson scheme
with the following preconditioning matrix:
\begin{equation}
	\label{precond}
	P=
	\begin{bmatrix}
		\Gamma               & -\lambda \tilde{R}_1^T \\
		-\lambda \tilde{R}_1 & -\lambda \tilde{R}_0
	\end{bmatrix}
\end{equation}
Solving a large linear system involving matrix $P$ is indeed solving
$m$ independent linear systems of dimensions $2n\times 2n$.

Notice that we have approximated $\tilde{\Psi}^TQ\tilde{\Psi}$ by keeping its
block-diagonal pattern. On the whole monolithic system \ref{mono} though, we
are not performing a block diagonal approximation: we are in fact keeping the
terms outside of the main diagonal involving $\tilde{R}_1$ matrices.

\section{Generalized cross validation}
Problem \ref{mono} is solved multiple times on a grid of $\lambda s$. The best
$\lambda$ is then chosen according to a generalized cross-validation criterion.
In particular, as in \cite{sangalli1}, the minimum of the generalized
cross-validation (GCV) parameter is used as a model selection criterion. The
corresponding parameter in the mixed-effect model reads as follows:
\begin{equation}
	\label{GCV}
	GCV(\lambda) = \frac{1}{N\left(1-\left(q-p+mp+tr\left(S\right)\right)/N\right)^2}\norm{\bm{z} - \hat{\bm{z}}}^2
\end{equation}
where for $q$, $m$, $p$ we follow the notation of section
\ref{repar}, and S indicates the smoothing matrix, which is the matrix that
maps the observations vector $\bm{z}$ into the estimated spatial field
$\hat{\bm{f}}$ evaluated at the location of the observations $(\hat{\bm{f}}_N =
	\tilde{\Psi}\hat{\bm{f}} = S z)$. Its value is
\begin{equation}
	S = \tilde{\Psi} \left(\tilde{\Psi}^TQ\tilde{\Psi} + \lambda \tilde{R}_1^T
	\tilde{R}_0^{-1} \tilde{R}_1\right)^{-1}\tilde{\Psi}^TQ,
\end{equation}
and it stems as a Schur complement for system \ref{mono} with respect
to the estimated field $\hat{\bm{f}}$.\\ The computation of the GCV parameter
\ref{GCV} involves the expensive computation of the trace of the smoothing
matrix $S$. Since for any two matrices $A$ and $B$ such that the product $AB$
is defined, it holds that $\trace \left(AB\right) = \trace \left(BA\right)$,
the trace of $S$ can be expressed as the trace of
\begin{equation}
	\label{s_tilda}
	\tilde{S} = \left(\tilde{\Psi}^TQ\tilde{\Psi} + \lambda \tilde{R}_1^T
	\tilde{R}_0^{-1} \tilde{R}_1\right)^{-1}\tilde{\Psi}^TQ \tilde{\Psi}.
\end{equation}
Several approaches are possible here for this computation.
\begin{enumerate}
	\item In an exact way, through Woodbury decomposition. Woodbury
	      decomposition, see \ref{sec:wood}, can be exploited here too
	      for the matrix inside the parenthesis. By choosing the following matrices
	      \begin{equation}
		      E = \tilde{\Psi}^T\tilde{\Psi} +  \lambda \tilde{R}_1^T
		      \tilde{R}_0^{-1} \tilde{R}_1
		      \quad U =
		      \tilde{\Psi}^TX
	      \end{equation}
	      \begin{equation*}
		      C = - \left( X^TX \right)^{-1}
		      \quad V = X^T \tilde{\Psi},
	      \end{equation*} $E$ is a block diagonal matrix that allows relatively easy system
	      solving. Other than $E$, with this approach we move the complex system to solve
	      to the dimensions of $X^TX$, that are $q-p+mp$. One obtains that
	      \begin{equation}
		      \trace \left(\tilde{S} \right) =
		      \trace \left(  \left( E^{-1} - E^{-1}U\left(C^{-1} + V E^{-1}U\right)^{-1}V E^{-1} \right)
		      \tilde{\Psi}^TQ \tilde{\Psi} \right).
	      \end{equation}
	      Rearranging, substituting, defining $G=E^{-1}\tilde{\Psi}^TQ
		      \tilde{\Psi}$, we get
	      \begin{equation}
		      \trace \left(\tilde{S} \right) =
		      \trace \left(G\right)  -
		      \trace \left(  E^{-1}\tilde{\Psi}^TX\left(X^T \tilde{\Psi} E^{-1}\tilde{\Psi}^TX - X^TX\right)^{-1}X^T \tilde{\Psi} G\right).
	      \end{equation}
	\item As an approximation, for convenience. In the library was first implemented a simplified method to
	      compute the degrees of freedom. It follows a logic similar to the one of the
	      iterative method. In the cases of chapter \ref{simul}, the difference with the exact method
	      was basically negligible, but no further analysis has been made.

	      In this method we approximate $\tilde{S}$ in equation \ref{s_tilda} by
	      considering the block diagonal approximation of the group
	      $\tilde{\Psi}^TQ\tilde{\Psi}$. As we have seen before, its $k$-th diagonal
	      block is $\Psi_i^TQ_i\Psi_i$. Trace of $\tilde{S}$ can therefore be written as
	      \begin{equation}
		      \trace \left(\tilde{S} \right) = \sum_{i=1}^{m} \left( \Psi^T_i Q_i\Psi_i  +  \lambda R_1^T R_0^{-1} R_1\right)^{-1}   \Psi_i^TQ_i\Psi_i .
	      \end{equation}

	\item[3-4.] As an estimation (possible both for method 1-2) deriving from the following result.
		\begin{theorem}[Hutchinson]
			Let $S$ be a symmetric matrix and let $\bm{u} =
				\begin{bmatrix}
					u_1 & \dots & u_N
				\end{bmatrix}
			$ be a vector of $N$ independent samples from a random variable $U$
			which takes values $-1$ and $1$ each with probability $1/2$. Then
			$\mathbb{E}[\bm{u}^TS\bm{u}] = \trace (S)$.
		\end{theorem}
		Matrix $\tilde{S}$ is symmetric, we can therefore exploit the theorem.
		Results are available in the literature on the asympotic properties of this
		estimator, which most importantly they do not depend on the size of the matrix.
		In practice, one makes a compromise between the speed of execution and the
		accuracy of the computation of the degrees of freedom. For our purposes, 60
		realizations of the random vector $\bm{u}$ were sufficient (tested on the
		C-shape domain, the grid of $\lambda$ was leading to larger variations).

		This method allows to reduce drastically the memory necessary to compute the
		trace of this matrix. Consider for example just the last matrix multiplication,
		$Q_i \Psi_i$. The result is a dense matrix because $Q_i$ is dense and will have
		dimensions $n_i \times n_{\tau_i}$, which could be already a large number, if
		the mesh is fine on a manifold or a 3D domain. By using previous theorem
		instead, $\Psi_i \bm{u}$ has length $n_i$, which multiplied by $Q_i$ is again a
		vector of length $n_i$.
\end{enumerate}
