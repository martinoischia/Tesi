% !TEX root = ../thesis.tex

\chapter{Solving large linear systems for the mixed-effects model}
\label{ch:chapter_name}
\section{Iterative methods}

Following the ideas stemmed from the spatio-temporal regression in
\citeauthor{pollini} \cite{pollini} and \citeauthor{massardi} \cite{massardi},
we consider an iterative scheme. At each step $i$ the scheme computes an
approximate solution $(\hat{\bm{f}}^{i},\hat{\bm{g}}^{i})$ of the monolithic
system by solving a single-unit problem for each statistical unit. As every
iterative scheme, we have (initialization equivalent to initializing to 0)....
The algorithm stops with two possible criteria:
\begin{itemize}
	\item[--]A maximum number of iterations is reached; \item[--] The following two
		conditions are true. \\ The first one is that the functional \ref{functional}
		evaluated in the estimated solution $J^i = J_{\Omega_i, \lambda}\left(
			\hat{\bm{\beta}}^{\prime i}, \hat{\bm{b}}^{\prime i}_1, \dots,
			\hat{\bm{b}}^{\prime i}_m, \hat{f}_1^i, \dots, \hat{f}_m^i \right)$ has reached
		stagnation, that is the relative increment $\left( J^i - J^{i-1}\right) / J^i$
		is below a certain threshold. In the code such threshold is an input parameter,
		it was set to $10^{-8}$. The term $ \int_{\Omega_i} \Delta f_i
			\left(\bm{p}\right)^2 d\Omega_i $ of the functional \ref{functional} is easily
		computed by exploiting the expansion into the finite element basis functions:
		it is equal to $\hat{\bm{g}}_i^T R_0 \hat{\bm{g}}_i$, where $\hat{\bm{g}}_i $
		is the sub-vector of $\hat{\bm{g}}$ corresponding to unit $i$.\\ The second one
		is that the estimated solution is close to the exact solution of the system
		\ref{mono}. This condition is verified by checking that the residual,
		normalized by the Euclidean norm of the right-hand side, is below a certain
		threshold (this is another input parameter, $10^{-8}$ was used).
\end{itemize}
The details of the method that was implemented are described in the
following sections.

\subsection{The block diagonal approximation}
\label[subsection]{iter}

Looking at the monolithic equation \ref{mono}, the question of how to formulate
an approximation of the term $\tilde{\Psi}^TQ\tilde{\Psi}$ naturally arises. In
particular, a suitable block approximation allows to make the estimated field
$\hat{f}_i$ of statistical unit $i$ independent of the observations in the
other units, for every unit $i$ ($i = 1 \dots m$). To this purpose, the
following approximation is considered:

\begin{equation}
	\label{blockdia}
	\tilde{\Psi}^TQ\tilde{\Psi}\simeq \Gamma :=
	\begin{bmatrix}
		\Psi^TQ_1\Psi & 0             & \dots  & 0             \\
		0             & \Psi^TQ_2\Psi & \ddots & \vdots        \\
		\vdots        & \ddots        & \ddots & 0             \\
		0             & \dots         & 0      & \Psi^TQ_m\Psi \\
	\end{bmatrix}
	,
\end{equation}
where $Q_i$ indicates the $i$-th diagonal block of $Q$, of dimensions
$n \times n$. By definition of $Q$, this is equal to
$I_n-X_i\left(X^TX\right)^{-1}X_i^T$, with $X_i=X\left(\left(i-1\right)n+1 :
	in\,, \;:\; \right)$, where the typical notation of MATLAB language has been
used to express a suitable submatrix.\\ This choice is not casual but it
naturally stems from the idea behind the iterative method as we will see in
section \ref{iterburocratico}. Not only: $\Psi^TQ_i\Psi$ is also the $i$-th
diagonal block of matrix $\tilde{\Psi}^TQ\tilde{\Psi}$. We will show it, for
example, by means of Woodbury decomposition (appendix \ref{sec:wood}) as used
in \ref{eq:wootilde}.

Express $\tilde{\Psi}^TQ\tilde{\Psi}$ as
$\tilde{\Psi}^T\tilde{\Psi}-\tilde{\Psi}^TH\tilde{\Psi}$. Thus, defining $U =
	\tilde{\Psi}^TX$, $C = -\left(X^TX\right)^{-1}$,$V =U^T$, its $i$-th diagonal
block is equal to $\Psi^T\Psi + U_iCV_i $, where
\begin{equation}
	\begin{split}
		U_i = U\left(\left(i-1\right)N_\mathcal{T}+1 : iN_\mathcal{T}\,, \;:\;\right) = \tilde{\Psi}^T_iX =\\
		\begin{bmatrix}
			0 & \dots & 0 & \Psi^T & 0 & \dots & 0
		\end{bmatrix}
		\begin{bmatrix}
			X_1    \\
			\vdots \\
			X_i    \\
			\vdots \\
			X_m
		\end{bmatrix}
		= \Psi^T X_i,
	\end{split}
\end{equation}
\begin{equation}
	V_i = U_i^T = X^T_i\Psi.
\end{equation}
Thus, obtaining what we wanted to prove,
\begin{equation}
	\Psi^T\Psi +  U_iCV_i = \Psi^T \left( I_n-X_i\left(X^TX\right)^{-1}X_i^T \right) \Psi = \Psi^T Q_i \Psi.
\end{equation}

\subsection{Initialization}
The initialization consists in finding a good guess
$(\hat{\bm{f}}^0,\hat{\bm{g}}^0)$ to start the algorithm from; for this
purpose, the following $m$ problems are solved: for $i = 1, \dots, m$ solve
\begin{equation}
	\begin{bmatrix}
		\Psi^TQ_i\Psi & -\lambda R_1^T \\
		-\lambda R_1  & -\lambda R_0
	\end{bmatrix}
	\begin{bmatrix}
		\hat{\bm{f}}_i^0 \\
		\hat{\bm{g}}_i^0
	\end{bmatrix}
	=
	\begin{bmatrix}
		\bm{u}_i \\
		0
	\end{bmatrix}
	,
\end{equation}
where $\bm{u}_i$ is the vector whose components are the first $n$
components of $\tilde{\Psi}^T Q\bm{z}$ starting from the $n(i-1) +1$-th
component.

\subsection{Iterations}
\label{iterburocratico}
The idea behind the iterative scheme is, having a guess of a solution
$(\hat{\bm{f}}^{k-1},\hat{\bm{g}}^{k-1})$, to compute a new guess
$(\hat{\bm{f}}^k,\hat{\bm{g}}^k)$ by replacing inside the monolithic system
\ref{mono} the unknown coefficients relative to all but one unit, with their
value computed at previous step, for every unit.

For example, the first $n$ equations of the monolithic system \ref{mono} read
\begin{equation}
	\begin{bmatrix}
		\Psi^TQ_{1,1}\Psi & \Psi^TQ_{1,2}\Psi & \dots & \Psi^TQ_{1,m}\Psi & -\lambda R_1^T \\
	\end{bmatrix}
	\begin{bmatrix}
		\hat{\bm{f}}_1 \\
		\hat{\bm{f}}_2 \\
		\vdots         \\
		\hat{\bm{f}}_m \\
		\hat{\bm{g}}_1
	\end{bmatrix}
	=
	\begin{bmatrix}
		\bm{u}_1
	\end{bmatrix}
	,
\end{equation}
where $Q_{i,j}$ indicates block of row $i$ and column $j$ of matrix
$Q$. Substituting $\hat{\bm{f}}_j$ for $j \neq 1$ with $\hat{\bm{f}}_j^{k-1}$,
taking previous step values to the right-hand side and generalizing, the
iterative scheme reads: for $i = 1, \dots, m$ solve
\begin{equation}
	\begin{bmatrix}
		\Psi^TQ_i\Psi & -\lambda R_1^T \\
		-\lambda R_1  & -\lambda R_0
	\end{bmatrix}
	\begin{bmatrix}
		\hat{\bm{f}}_i^k \\
		\hat{\bm{g}}_i^k
	\end{bmatrix}
	=
	\begin{bmatrix}
		\bm{r}_i \\
		0
	\end{bmatrix}
\end{equation}
where
\begin{equation}
	\bm{r}_i =\bm{u}_i -\sum_{\substack{j=1\\ j\neq i}}^m \Psi^TQ_{i,j} \Psi \hat{\bm{f}}_j^{k-1}
\end{equation}

An estimate of $\bm{\nu}$, $\hat{\bm{\nu}}^i = (\hat{\bm{\beta}}^{\prime i},
	\hat{\bm{b}}_1^{\prime i}, \dots, \hat{\bm{b}}_m^{\prime i})$ has to be
computed at each iteration for the estimation of the functional
\ref{functional}, according to equation \ref{nu}.

\subsection{The iterative method as a preconditioned Richardson method}
Given a generic preconditioning matrix $P$, defining $\bm{r}_k=A \bm{x}_k -
	\bm{b}$ the residual of the linear system $A \bm{x} = \bm{b}$ at step $k$,
Richardson method (\textit{cf.}, for example, \cite{Quarteroni}) consists in
solving (or rather trying to) the linear system iterating the following steps:
\begin{enumerate}
	\item Solve $P \bm{z}_k = \bm{r}_k$
	\item Compute the acceleration parameter $\alpha_k$ (for simplicity we use $\alpha_k = 1$)
	\item Update the solution $ \bm{x}_{k+1} = \bm{x}_{k} - \alpha_k \bm{z}_k $
	\item Update the residual $\bm{r}_{k+1} = \bm{r}_{k} - \alpha_k A \bm{z}_k$
\end{enumerate}

The iterative scheme described in the previous section is a Richardson scheme
with the following preconditioning matrix:
\begin{equation}
	\label{precond}
	P=
	\begin{bmatrix}
		\Gamma               & -\lambda \tilde{R}_1^T \\
		-\lambda \tilde{R}_1 & -\lambda \tilde{R}_0
	\end{bmatrix}
\end{equation}
Solving a large linear system involving matrix $P$ is indeed solving
$m$ independent linear systems of dimensions $2n\times 2n$.
\section{Generalized cross validation}
Problem \ref{mono} is solved multiple times on a grid of $\lambda s$. The best
$\lambda$ is then chosen according to a generalized cross-validation criterion.
In particular, as in \cite{sangalli1}, the minimum of the generalized
cross-validation (GCV) parameter is used as a model selection criterion. The
corresponding parameter in the mixed-effect model reads as follows:
\begin{equation}
	\label{GCV}
	GCV(\lambda) = \frac{1}{N\left(1-\left(q-p+mp+tr\left(S\right)\right)/N\right)^2}\norm{\bm{z} - \hat{\bm{z}}}^2
\end{equation}
where for $q$, $m$, $p$ we follow the notation of section
\ref{repar}, and S indicates the smoothing matrix, which is the matrix that
maps the observations vector $\bm{z}$ into the estimated spatial field
$\hat{\bm{f}}$ evaluated at the location of the observations $(\hat{\bm{f}}_N =
	\tilde{\Psi}\hat{\bm{f}} = S z)$. Its value is
\begin{equation}
	S = \tilde{\Psi} \left(\tilde{\Psi}^TQ\tilde{\Psi} + \lambda \tilde{R}_1^T
	\tilde{R}_0^{-1} \tilde{R}_1\right)^{-1}\tilde{\Psi}^TQ,
\end{equation}
and it stems as a Schur complement for system \ref{mono} with respect
to the estimated field $\hat{\bm{f}}$.\\ The computation of the GCV parameter
\ref{GCV} involves the expensive computation of the trace of the smoothing
matrix $S$. Since for any two matrices $A$ and $B$ such that the product $AB$
is defined, it holds that $\trace \left(AB\right) = \trace \left(BA\right)$,
the trace of $S$ can be computed as the trace of
\begin{equation}
	\label{s_tilda}
	\tilde{S} = \left(\tilde{\Psi}^TQ\tilde{\Psi} + \lambda \tilde{R}_1^T
	\tilde{R}_0^{-1} \tilde{R}_1\right)^{-1}\tilde{\Psi}^TQ \tilde{\Psi}.
\end{equation}
Woodbury decomposition, see \ref{sec:wood}, can be exploited here too
for the matrix inside the parenthesis. By choosing the following matrices
\begin{equation}
	E = \tilde{\Psi}^T\tilde{\Psi} +  \lambda \tilde{R}_1^T
	\tilde{R}_0^{-1} \tilde{R}_1
	\quad U =
	\tilde{\Psi}^TX
\end{equation}
\begin{equation*}
	C = - \left( X^TX \right)^{-1}
	\quad V = X^T \tilde{\Psi},
\end{equation*} $E$ is a block diagonal matrix that allows easy system solving.
Other than $E$, with this approach we move the complex system to solve to the
dimensions of $X^TX$, that are $q-p+mp$. One obtains that
\begin{equation}
	\trace \left(\tilde{S} \right) =
	\trace \left(  \left( E^{-1} - E^{-1}U\left(C^{-1} + V E^{-1}U\right)^{-1}V E^{-1} \right)
	\tilde{\Psi}^TQ \tilde{\Psi} \right).
\end{equation}
Rearranging, substituting, defining $G=E^{-1}\tilde{\Psi}^TQ
	\tilde{\Psi}$, we get
\begin{equation}
	\trace \left(\tilde{S} \right) =
	\trace \left(G\right)  -
	\trace \left(  E^{-1}\tilde{\Psi}^TX\left(X^T \tilde{\Psi} E^{-1}\tilde{\Psi}^TX - X^TX\right)^{-1}X^T \tilde{\Psi} G\right).
\end{equation}

For convenience, in the library was first implemented a simplified method to
compute the degrees of freedom, while the method just described is left to be
programmed. The implemented method follows a logic similar to the one of the
iterative method. In the simulations and the applications described later it
did not constitute an issue, and where was possible to compute the value in
different ways we notice the difference was basically negligible.

In this method we approximate $\tilde{S}$ in equation \ref{s_tilda} by
considering the block diagonal approximation of the group
$\tilde{\Psi}^TQ\tilde{\Psi}$. As we have seen before, its $k$-th diagonal
block is $\Psi^TQ_i\Psi$. Trace of $\tilde{S}$ can therefore be written as
\begin{equation}
	\trace \left(\tilde{S} \right) = \sum_{i=1}^{m} \left( \Psi^TQ_i\Psi  +  \lambda \tilde{R}_1^T \tilde{R}_0^{-1} \tilde{R}_1\right)^{-1}   \Psi^TQ_i\Psi .
\end{equation}
