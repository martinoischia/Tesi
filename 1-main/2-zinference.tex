% !TEX root = ../thesis.tex
\chapter{Statistical inference}
\section{Properties of the estimators}
\subsection{Spatial field component}
\nocite{*}
To study the statistical properties of the estimated spatial field that is obtained by solving the monolithic system \ref{mono}, we write it in the following equivalent manner
\begin{equation}
	\begin{cases}
		\tilde{\Psi}^TQ\tilde{\Psi} \hat{\bm{f}} -\lambda \tilde{R}_1^T \hat{\bm{g}} = \tilde{\Psi}^T Q\mathbf{z} \\
		\hat{\bm{g}} = - \tilde{R}_0^{-1} \tilde{R}_1 \hat{\bm{f}}
	\end{cases}
\end{equation}
By recalling the mixed-effect model in the implementative form
\ref{modelX}, that is $\bm{z} = X \bm{\nu} + \bm{f}_N + \bm{\epsilon}$, and
substituting the second equation inside the first we get
\begin{equation}
	\tilde{\Psi}^TQ\tilde{\Psi} \hat{\bm{f}} +\lambda \tilde{R}_1^T \tilde{R}_0^{-1} \tilde{R}_1 \hat{\bm{f}} = \tilde{\Psi}^T Q \left( X \bm{\nu} + \bm{f}_N + \bm{\epsilon} \right)
\end{equation}
Recalling that Q projects on the orthogonal space of the columns of
X, implying $Q X = 0$,
\begin{equation}
	\left(\tilde{\Psi}^TQ\tilde{\Psi} +\lambda \tilde{R}_1^T \tilde{R}_0^{-1} \tilde{R}_1 \right) \hat{\bm{f}} = \tilde{\Psi}^T Q \bm{f}_N + \tilde{\Psi}^T Q \bm{\epsilon}
\end{equation}
That can be rearranged as follows
\begin{equation}
	\left(\tilde{\Psi}^TQ\tilde{\Psi} +\lambda \tilde{R}_1^T \tilde{R}_0^{-1} \tilde{R}_1 \right)
	\left(\hat{\bm{f}} - \bm{f}\right)
	+ \lambda \tilde{R}_1^T \tilde{R}_0^{-1} \tilde{R}_1 \bm{f} = \tilde{\Psi}^T Q \bm{\epsilon}
\end{equation}
where we have defined $\bm{f}$ as the coefficients of the finite
element expansion of the true vector field $f(\cdot)$. Deriving now the term
$\hat{\bm{f}} - \bm{f}$
\begin{equation}
	\hat{\bm{f}} - \bm{f} =
	-\left(\tilde{\Psi}^TQ\tilde{\Psi} +\lambda \tilde{R}_1^T \tilde{R}_0^{-1} \tilde{R}_1 \right)^{-1}
	\lambda \tilde{R}_1^T \tilde{R}_0^{-1} \tilde{R}_1 \bm{f} +
	\left(\tilde{\Psi}^TQ\tilde{\Psi} +\lambda \tilde{R}_1^T \tilde{R}_0^{-1} \tilde{R}_1 \right)^{-1}
	\tilde{\Psi}^T Q \bm{\epsilon}
\end{equation}
The first term above is not random, thus we can derive the following:
\begin{enumerate}[wide, labelindent=0pt]
	\item Thanks to the property that, for generic suitably dimensioned matrices and vectors, $\mathbb{E}\left[M \bm{x}\right] = M \mathbb{E}\left[\bm{x}\right]$, we get that $\mathbb{E}\left[\hat{\bm{f}} - \bm{f}\right] =
		      -\left(\tilde{\Psi}^TQ\tilde{\Psi} +\lambda \tilde{R}_1^T \tilde{R}_0^{-1} \tilde{R}_1 \right)^{-1}
		      \lambda \tilde{R}_1^T \tilde{R}_0^{-1} \tilde{R}_1 \bm{f}$;
	\item Thanks to the property that again, for generic suitably dimensioned matrices and vectors, $Var  \left[M \bm{x}\right] = M Var\left[\bm{x}\right] M^T$, we get this result for $Var\left[\hat{\bm{f}} - \bm{f}\right] = Var \left[\hat{\bm{f}}\right]$
	      \begin{equation}
		      Var\left[\hat{\bm{f}} - \bm{f}\right] =
		      \sigma^2 \left(\tilde{\Psi}^TQ\tilde{\Psi} +\lambda \tilde{R}_1^T \tilde{R}_0^{-1} \tilde{R}_1 \right)^{-1} \tilde{\Psi}^T Q
		      \tilde{\Psi} \left(\tilde{\Psi}^TQ\tilde{\Psi} +\lambda \tilde{R}_1^T \tilde{R}_0^{-1} \tilde{R}_1 \right)^{-1}
	      \end{equation}
	      where we have exploited symmetric matrices and the fact that $Q^2 =
		      Q$.
\end{enumerate}
\subsection{Parametric component}
For what concerns the estimation of the parameter $\bm{\nu}$, we recall that
$\hat{\bm{\nu}}$ solves the normal equations
\begin{equation}
	X^T X \hat{\bm{\nu}} = X^T \left( \bm{z} - \tilde{\Psi} \hat{\bm{f}} \right)
\end{equation}
that with model \ref{modelX} can be written as
\begin{equation}
	X^T X \hat{\bm{\nu}} =
	X^T \left( X \bm{\nu} + \tilde{\Psi} \bm{f} + \bm{\epsilon}
	- \tilde{\Psi} \hat{\bm{f}} \right)
\end{equation}
or
\begin{equation}
	X^T X \left(\hat{\bm{\nu}} - \bm{\nu}\right)= -
	X^T \tilde{\Psi} \left( \hat{\bm{f}} - \bm{f} \right)
	+ X^T \bm{\epsilon}
\end{equation}
From this we can derive:
\begin{equation}
	\mathbb{E}\left[\hat{\bm{\nu}} - \bm{\nu}\right] =
	\left(X^T X\right)^{-1} X^T
	\tilde{\Psi}\left(\tilde{\Psi}^TQ\tilde{\Psi} +\lambda \tilde{R}_1^T \tilde{R}_0^{-1} \tilde{R}_1 \right)^{-1}
	\lambda \tilde{R}_1^T \tilde{R}_0^{-1} \tilde{R}_1 \bm{f}
\end{equation}
and the expression of $Var \left[\hat{\bm{\nu}} - \bm{\nu}\right]$
that is not reported for brevity.

%\subsection{Asymptotic properties}
%We study the limiting properties of the estimators in the setting where the number of basis $N_\mathcal{T}$ and the triangulation $\mathcal{T}$ are fixed. The number of observation $n_i$ for each statistical unit increases to infinity.
%
%In a similar fashion as in \cite{sangalli2}, where the simple spatial problem is considered, we make some assumptions for studying the asymptotic properties of the considered estimators.
