% !TEX root = ../thesis.tex

\addchap{Sommario}
In questo lavoro viene proposto un algoritmo iterativo per sistemi
lineari di grandi dimensioni e viene implementato nella libreria di
R \texttt{fdaPDE}.
Le equazioni originano da un problema di minimizzazione di un
problema funzionale che viene discretizzato attraverso gli elementi
finiti. Infatti, nel contesto di un modello a effetti misti, ogni
unità statistica aumenta la dimensione del sistema di $2 N_{\tau}$, 
due volte il numero di gradi di libertà degli elementi finiti.\newline
Per rendere possibile il calcolo di una soluzione, ad ogni iterazione
vengono risolti $m$ sistemi lineari di dimensioni
$\left(2 N_{\tau}\right)^2$, con $m$ il numero di unità statistiche,
al posto di uno di dimensioni $\left(2m N_{\tau}\right)^2$.\newline
Vengono condotte delle simulazioni al calcolatore e il modello proposto
viene infine applicato a dati di risonanza magnetica funzionale nella
corteccia cerebrale, permettendo di aumentare il numero di pazienti
inclusi nello studio, rispetto agli studi esistenti.

